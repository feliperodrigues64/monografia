% Nome do Anexo
\chapter{Primeiro Anexo}
\label{primeiro-anexo}
% Para diminuir espa�amento entre o t�tulo e o texto
\vspace{-1.9cm}

% Texto
% Textos ou documentos não elaborados pelo autor.

\lstinputlisting[
  language=c, 
  label=leitura-arquivos-diretorio-node, 
  caption={Fonte: \cite{pereira}. O inferno em chamadas de retorno}
]{pos-texto/leitura-arquivos-diretorio-node.js}

\lstinputlisting[
  language=c, 
  label=leitura-arquivos-diretorio-node-callback-heaven, 
  caption={Fonte: \cite{pereira}. Chamadas de retorno organizada}
]{pos-texto/leitura-arquivos-diretorio-node-callback-heaven.js}

\lstinputlisting[
  language=c, 
  label=leitura-arquivos-diretorio-node-strongloop, 
  caption={Fonte: \cite{strongloop}. Exemplo complexo de leitura de arquivos com callbacks hell}
]{pos-texto/leitura-arquivos-diretorio-node-strongloop.js}

\lstinputlisting[
  language=c, 
  label=leitura-arquivos-diretorio-node-strongloop-modular, 
  caption={Fonte: \cite{strongloop}. Exemplo complexo de leitura de arquivos modularizado}
]{pos-texto/leitura-arquivos-diretorio-node-strongloop-modular.js}

\lstinputlisting[
  language=c, 
  label=node-js-app, 
  caption={Fonte: Autor. Arquivo central para aplicação Node}
]{pos-texto/app.js}

\lstinputlisting[
  language=c,
  label=node-js-contact-route, 
  caption={Fonte: Autor. Módulo para a API de contato}
]{pos-texto/contacts.js}
