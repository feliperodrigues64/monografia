% Resumo
\begin{resumo}
% Diminuir espaçamento entre título e texto
\vspace{-1cm}

% Texto do resumo: sem paragrafo, justificado, com espaçamento 1,5 cm
\onehalfspacing

\noindent
  
  O crescente número de usuários conectados a aplicações nos servidores contribuiu
  para  a mudança no perfil dos desenvolvedores, que agora devem pensar na arquitetura web 
  para oferecer o serviço a milhões de usuários. Inúmeras técnicas foram desenvolvidas 
  com o passar dos anos para resolver essa demanda, como técnicas de cache e escalonamento de servidores.
  Uma dessas técnicas foi a implementação do Node.Js como plataforma capaz de atender esses requisitos
  e oferecer ao desenvolvedor um ambiente que resolva esses pontos criticos. Motivada por estes desafios
  essa pesquisa investigativa sobre desenvolvimento e desempenho de aplicações web propõe um estudo da plataforma
  Node.Js e suas características capazes de alcançar alto desempenho em servidores web. Após estudar o Node.Js 
  é iniciado o desenvolvimento de dois protótipos um em ambiente Django e outro em ambiente Node.Js. 
  Ao concluir o desenvolvimento dos protótipos, foi realizada uma comparação com testes
  de carga para medir o desempenho de cada arquitetura e tecnologia utilizada. Como conclusão e comparação dos testes 
  foi observado que com o Node.Js obtiveram melhores resultados. 
  

% Espaçamento para as palavras-chave
\vspace*{.75cm}

% Palavras-chave: sem parágrafo, alinhado à esquerda
\noindent Palavras-chave: Node.Js; programação orientada a eventos; Análise de desempenho.\\
% Segunda linha de palavras-chave, com espaçamento.
%\indent\hspace{2cm}Palavra.

\end{resumo}