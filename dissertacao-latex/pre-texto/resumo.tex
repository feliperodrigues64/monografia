% Resumo
\begin{resumo}
% Diminuir espaçamento entre título e texto
\vspace{-1cm}

% Texto do resumo: sem paragrafo, justificado, com espaçamento 1,5 cm
\onehalfspacing

\noindent
  
  
  Com o crescente número de usuários conectados a aplicações nos servidores,
  é necessário repensar no modelo de desenvolvimento de aplicações web existentes atualmente.
  Na atual arquitetura web tem-se de pensar na arquitetura do sistema, utilizar técnicas de cache e 
  investir em escalonamento de servidores.  
  
  O Node.Js propõe alterar esse paradigma com o objetivo de fornecer uma arquitetura de software 
  capaz de receber milhões de conexões simultâneas e ser facilmente escalável, 
  sem necessidade de custos exorbitantes com infraestrutura.
  
  Para validar a potencialidade do Node.Js vamos estudar o básico deste ambiente e em seguida realizar um
  estudo de caso com uma API REST, muito utilizado no serviços web. Após o estudo será desenvolvido
  dois sistemas onde iremos comparar a forma de desenvolvimento de um \textit{framework} Django e 
  do \textir{framework} Express.js para Node.Js
  
  Com os aplicativos desenvolvidos serão realizados testes de carga nos dois ambientes verificando o tempo
  de resposta e sucessos em cada requisições. E por fim vamos analisar os resultados dos testes e informar
  ao leitor a capacidade de cada sistema.

% Espaçamento para as palavras-chave
\vspace*{.75cm}

% Palavras-chave: sem parágrafo, alinhado à esquerda
\noindent Palavras-chave: Node.Js; programação orientada a eventos; Análise de desempenho.\\
% Segunda linha de palavras-chave, com espaçamento.
%\indent\hspace{2cm}Palavra.

\end{resumo}