% Modelo de Dissertação do Mestrado em Informática da PUC - Alterado para cumprir a normalização de 2011
%\documentclass[a4paper,brazil,ruledheader,normaltoc,capchap]{abnt}

% Para impressão frente e verso (normalização 2011)
\documentclass[a4paper,brazil,ruledheader,normaltoc,capchap,twoside,openany]{abnt_pucmg_utf8}

% Não esquecer das alterações no arquivo abnt.cls
% Se estiver usando o Kile no Ubuntu o arquivo fica armazenado em /usr/share/texmf/tex/latex/abntex.
% Comentar a linha 967 
% \vspace*{30pt}% - Linha comentada para reduzir o espaçamento entre o topo da página e o título \chapter
% Alterar a linha 1143
% \vspace*{-30pt} % - Parâmetro alterado de 30pt para -30pt para reduzir o espaçamento entre o top da página e o título do apêndice
% Alterar a linha 985
%\vspace*{-30pt}\par % - Parâmetro alterado de 0pt para -30pt para reduzir o espaçamento entre o top da página e o título \chapter*
% Alterar a linha 991
% Parâmetro alterado de 45pt para 30pt para reduzir o espaçamento entre o texto e o título \chapter*

% Não esquecer das alterações no arquivo acronym.sty
% Se estiver usando o Kile no Ubuntu o arquivo fica armazenado em /usr/share/texmf-texlive/tex/latex/acronym
% Alterar a linha 225
%\item[\protect\AC@hypertarget{#1}{\acsfont{\normalfont{#2}}} --] #3% - Inserir separador entre acrônimo/descrição e remover o negrito com o normalfont

% Pacote para definir explicitamente as margens das páginas
\usepackage[a4paper,left=3cm,right=2cm,top=3cm,bottom=2cm]{geometry}

%\usepackage{units}
% Utilize da seguinte forma \unit[78,6]{mA}

% Pacote para gerenciar siglas
\usepackage[printonlyused]{acronym}

% Merge em duas células (linhas diferentes)
\usepackage{multirow}

% Pacote para citação e referências seguindo ABNT no sistema (AUTOR, Data)
%\usepackage[alf, bibjustif,abnt-emphasize=bf]{abntcite}
\usepackage[alf, abnt-emphasize=em, abnt-thesis-year=title]{abntcite}
% @article An article from a journal or magazine 
% @inproceedings An article in a conference proceedings
% Força que o tipo de ênfase no nome do simpósio seja em caixa alta
\renewcommand{\emph}{\textsc}

% Pacote para múltiplos arquivos .bib
\usepackage{multibib}
%\newcites{pub}{Refer\^encias das publica\c{c}\~oes}

% Pacote de adequação do formato ABNT para normas da PUCMinas
\usepackage{abnt-PPGInf-PUCMG}

% Pacotes utilitários
\usepackage{graphicx}

% Pacote para fixar a figura no local desejado
\usepackage{float}

% Pacote para adicionar simbolos as informações de rodapé
\usepackage[symbol]{footmisc}

\usepackage[all]{xy}
%\usepackage[tight]{subfigure}	% Permite a criação de subfiguras
\usepackage{url,amsmath}	% Permite melhorias na codificação de fórmulas
%\usepackage{amsthm}		% Permite melhorias na escrita de teoremas
\usepackage{amssymb}		% Permite utlização de simbolos matemáticos avançados

\usepackage[portuguese, linesnumbered, ruled, vlined]{algorithm2e}
\usepackage{algorithmic} 	% para algoritmos
\usepackage{listings} 		% para importação de código-fonte

% Alterar o espaçamento da margem no algoritmo
\setlength{\algomargin}{1em}

\usepackage{setspace}

% Pacote para rotação de tabelas/figuras
\usepackage{rotating}

% Pacotes para criação de cronograma/tabela colorida
\usepackage{color}
\usepackage{array}
\usepackage{longtable}
\usepackage{colortbl}
%\definecolor{lightgray}{gray}{0.9}

% Pacote para possibilitar o uso do setboolean para forçar formatos de página diferentes do padrão do documento
\usepackage{ifthen}

% Para inserir captions (nova normalização 2011)
\usepackage[size=normalsize,labelfont=bf,textfont={bf},labelsep=endash]{caption}
\captionsetup[subfloat]{labelfont=bf,textfont={bf}}

% Usado para reduzir espaçamentos entre itens (alíneas, enumerações) com o compactitem
\usepackage{paralist}

% Alterar para sequencial a numeração de figuras e tabelas
\captionsetup{figurewithin=none}
\captionsetup{tablewithin=none}

\setlength{\LTcapwidth}{\textwidth}

% Para o subsubsection aparecer no sumário 
\setcounter{tocdepth}{3}
\setcounter{secnumdepth}{3}

% Para inserir referências via links - não funciona para abntex
%\usepackage[colorlinks=true,pdfstartview=FitV,linkcolor=blue,citecolor=blue,urlcolor=blue,hyperindex,pagebackref=true,pdftex,breaklinks]{hyperref}
%\usepackage[pdftex]{hyperref}

% Para criar lista de gráficos
\floatstyle{plaintop}
\newfloat{grafico}{H}{loq}
\restylefloat*{grafico}
\floatname{grafico}{Gráfico} 

% Para gerar subfiguras usando o subfloat
\usepackage{subfig}
\newsubfloat[position=bottom,listofformat=subsimple]{grafico}

% define estilo de posicionamento na caixa
\newsavebox{\leftfig}
\newsavebox{\rightfig}

%\renewcommand{\ALG@name}{Algoritmo}
%\renewcommand{\listalgorithmname}{Lista de Algoritmos}

% Configuração de código-fonte
\lstset{extendedchars=\true, % permite acentos
 inputencoding=utf8,
 literate={\$}{{\$}}1,
 commentstyle=\it, % deixa os comentários em itálico
 stringstyle=\bf, % não lembro o que faz, mas está funcionando
 belowcaptionskip=5pt, % não lembro o que faz, mas está funcionando
 numbers=left, % coloca a numeração na esquerda
 stepnumber=1, % passos da numeração
 firstnumber=1, % primeira linha
 numberstyle=\tiny, % tamanho da fonte da numeração
 breaklines=true, % permitir quebra de linha
 frame=tb, % borda em cima e em baixo
 basicstyle=\footnotesize, % estilo básico
 stringstyle=\ttfamily, % não lembro o que faz, mas está funcionando
 showstringspaces=false, % não mostrar os espaços
 mathescape, % não lembro o que faz, mas está funcionando
 tabsize=3 % tamanho da tabulação
}
\renewcommand{\lstlistingname}{Código}
\renewcommand{\lstlistlistingname}{Lista de Códigos}
\citeoption{abnt-etal-cite=1, abnt-and-type=e}

% the bibtex style generates this command, but it's not defined
\newcommand{\optionaltextstyle}{}

%%\pdfinfo{%
%%  /Title    (TITULO DA DISSERTACAO)
%%  /Author   (Nome do aluno)
%%  /Creator  (Nome do aluno)
%%  /Producer (Kile - an Integrated LaTeX Environment - %%Version 2.0.85)
%%  /Subject  (Dissertação de Mestrado)
%% /Keywords (Palavras chave)
%%}

% PRÉ-TEXTUAIS %%
\begin{document}

% Para forçar que elementos pré-textuais (da capa até o sumário) sejam impressos no anverso da folha
\setboolean{@twoside}{false}

\autor{Lucas Simon Rodrigues Magalhães}

% Coloque o título em caixa alta. É o padrão da PUC.
% Vá no arquivo abnt-PPGInf-PUCMG.sty e procure por esse título (linha 575). Altere para o seu título em caixa alta. Isso será utilizado na folha de aprovação.
\titulo{Utilizando JavaScript no servidor para construir aplicações de alta concorrência na internet com Node.Js}

\orientador[Orientador:]{Prof. João Caram}

% Se não tiver, co-orientador, comente a próxima linha.
%\coorientador[Co-orientador:]{Professor}

% Texto
\comentario{Monografia apresentada ao Departamento de Sistemas de Informação da Pontifícia Universidade Católica de Minas Gerais,
como exigência parcial para a obtenção do título de Bacharel em Sistemas de Informação.}


% Instituição
\instituicao{Bacharelado em Sistemas de Informação \par Departamento de Sistemas de Informação \par Pontifícia Universidade Católica de Minas Gerais}

% Local
\local{Belo Horizonte}

% Data
\data{24 de Novembro de 2014}
\capa
%Para forçar que a ficha catalográfica seja impressa no verso da folha de aprovação
\setboolean{@twoside}{true}

% Gera a folha de rosto
\folhaderosto

% Ficha catalográfica
% Ficha catalográfica
% INCLUIR O ARQUIVO PDF GERADO PELA BIBLIOTECA COMO FIGURA.
\begin{figure}[h!]
	\vspace*{-3.3cm}
	\hspace*{-3cm}
%	% Suponha o nome do arquivo em pre-texto/ficha-catalografica/fichacatalografica.pdf
	\includegraphics{pre-texto/ficha-catalografica} 
	\newpage
\end{figure}

% Para forçar que elementos pré-textuais (da capa até o sumário) sejam impressos no anverso da folha
\setboolean{@twoside}{false}

% Folha de aprovação
% Termo de Aprovação

% Texto da aprovação
\textoaprovacao{Dissertacao apresentada ao Programa de Pos-Graduacao em Informatica como requisito parcial para qualificacao ao Grau de Mestre em Informatica pela Pontificia Universidade Catolica de Minas Gerais.}

% Primeira assinatura
\primeiroassina{Prof. Dr. Orientador -- PUC Minas}

% Segunda assinatura
\segundoassina{Prof.$^{a}$ Dr.$^{a}$ Membro interno -- Instituicao}

% Terceira assinatura
\terceiroassina{Prof. Dr. Membro externo -- Instituicao}

% Quarta assinatura
%\quartoassina{}

% Data da defesa
\localdia{Belo Horizonte, data da defesa.}

% Gera o termo de aprovação
\termodeaprovacao	

% Dedicatória
% Dedicatória
\newpage

% Espaçamento do topo da página até o texto da dedicatória
\vspace*{22cm}

% Espaçamento na esqueda
\hspace{8cm}\begin{minipage}{.60\textwidth}
            \textit{Dedico este trabalho à meus familiares e amigos que me acompanharam durante todo esse tempo em meus estudos, às vezes me apoiando e às vezes puxando minha orelha.}
            \end{minipage}

% Agradecimentos
% Agradecimentos
%\chapter*{Agradecimentos}
\begin{center}
	\normalsize
	\textbf{AGRADECIMENTOS}
\end{center}

  Aos meus pais, Ivam Silva Magalhães e Haydee Rodrigues Magalhães, por todo os valores, educação e estímulos aos meus estudos, profissionalmente e pessoalmente. A minha irmã Nadia Rodrigues Magalhães que nos momentos mais difíceis sempre me procurou, conversou e me apoiou de todas as formas possíveis.
  
  Ao meu amigo Igor Tadeu Camilo Rocha que me ajudou a escrever, corrigir e tornar deste um trabalho de sucesso

% Epígrafe
% Epígrafe
\newpage

% Espaçamento entre topo da página e texto da epígrafe
\vspace*{10cm}
% Espaçamento na esqueda
\hspace{4cm}\begin{minipage}{.51\textwidth}

% Texto da epígrafe
\textit{``E fazendo que se aprende a fazer aquilo que se deve aprender a fazer.'' }

%Nome do autor
\begin{flushright}\itshape Aristoteles \upshape\end{flushright}

\end{minipage}

% Resumo
% Resumo
\begin{resumo}
% Diminuir espaçamento entre título e texto
\vspace{-1cm}

% Texto do resumo: sem paragrafo, justificado, com espaçamento 1,5 cm
\onehalfspacing

\noindent
  
  
  Com o crescente número de usuários conectados a aplicações nos servidores,
  é necessário repensar no modelo de desenvolvimento de aplicações web existentes atualmente.
  Na atual arquitetura web tem-se de pensar na arquitetura do sistema, utilizar técnicas de cache e 
  investir em escalonamento de servidores.  
  
  O Node.Js propõe alterar esse paradigma com o objetivo de fornecer uma arquitetura de software 
  capaz de receber milhões de conexões simultâneas e ser facilmente escalável, 
  sem necessidade de custos exorbitantes com infraestrutura.
  
  Para validar a potencialidade do Node.Js vamos estudar o básico deste ambiente e em seguida realizar um
  estudo de caso com uma API REST, muito utilizado no serviços web. Após o estudo será desenvolvido
  dois sistemas onde iremos comparar a forma de desenvolvimento de um \textit{framework} Django e 
  do \textir{framework} Express.js para Node.Js
  
  Com os aplicativos desenvolvidos serão realizados testes de carga nos dois ambientes verificando o tempo
  de resposta e sucessos em cada requisições. E por fim vamos analisar os resultados dos testes e informar
  ao leitor a capacidade de cada sistema.

% Espaçamento para as palavras-chave
\vspace*{.75cm}

% Palavras-chave: sem parágrafo, alinhado à esquerda
\noindent Palavras-chave: Node.Js; programação orientada a eventos; Análise de desempenho.\\
% Segunda linha de palavras-chave, com espaçamento.
%\indent\hspace{2cm}Palavra.

\end{resumo}

% Abstract
% Abstract
\begin{abstract}
% Diminuir espaçamento entre título e texto
\vspace{-1cm}
% Texto do resumo, em inglês: sem paragrafo, justificado, com espaçamento 1,5 cm
\onehalfspacing
\noindent
  Texto do resumo, em ingles.

% Espaçamento para as palavras-chave
\vspace*{.75cm}

% Palavras-chave: sem parágrafo, alinhado à esquerda
\noindent Keywords: . \\
% Segunda linha de palavras-chave, com espaçamento.
%\indent\hspace{1.4cm} Keyword.

\end{abstract}

\makeatletter
\renewcommand\numberline[1]{
	\leftskip 0em
	\rightskip 1.6em
	\parfillskip -\rightskip
	\parindent 0em
	\@tempdima 2.0em
	\vspace{0em} \advance\leftskip \@tempdima \null\nobreak\hskip -\leftskip
	FIGURA \normalfont #1 -- }
\makeatother

% Lista de figuras
\listoffigures

\makeatletter
\renewcommand\numberline[1]{
	\leftskip 0em
	\rightskip 1.6em
	\parfillskip -\rightskip
	\parindent 0em
	\@tempdima 2.0em
	\vspace{0em} \advance\leftskip \@tempdima \null\nobreak\hskip -\leftskip
	TABELA \normalfont #1 -- }
\makeatother

% Lista de tabelas
\listoftables

\makeatletter
\renewcommand\numberline[1]{
	\leftskip 0em
	\rightskip 1.6em
	\parfillskip -\rightskip
	\parindent 0em
	\@tempdima 2.0em
	\vspace{0.5em} \advance\leftskip \@tempdima \null\nobreak\hskip -\leftskip
	GRÁFICO \normalfont #1 -- }
\makeatother

% Lista de graficos
\listof{grafico}{Lista de Graficos}

% Lista de siglas
% Lista de Abreviaturas e Siglas
%\chapter*{Lista de Abreviaturas e Siglas}
\chapter*{Lista de Abreviaturas e Siglas}

% Mantenha sempre em ordem alfabética.

\begin{acronym}
\acro{API} {Interface de Programação de Aplicativos, do inglês \textit{Application Programming Interface}}
\acro{CLI} {Interface para a linha de comando, do inglês \textit{Command Line Interface}}
\acro{CPU} {Unidade central de processamento, do inglês \textit{Central Processing Unit}}
\acro{CRUD} {Criar, ler, atualizar e deletar, do inglês \textit{Create, Read, Update, Destroy}}
\acro{DNS} {Sistema de nome de Domínios, do inglês \textit{Domain Name System}}
\acro{E/S} {\textit{Entrada e Saída}}
\acro{HTML} {Linguagem de marcação de hipertexto, do inglês \textit{HyperText mark-up Language}}
\acro{HTTP} {Protocolo de transferência de hipertexto, do inglês \textit{HyperText Transfer Protocol}}
\acro{JSON} {Notação de objetos Javascript, do inglês \textit{JavaScript Object Notation}}
\acro{MVR} {Modelo, visão e rotas, do inglês \textit{Model View Routes}}
\acro{MVC} {Modelo, visão e controladores, do inglês \textit{Model View Controller}}
\acro{NOSQL} {Não Somente SQL usado para definir bancos de dados não relacionais, do inglês \textit{Not Only SQL}}
\acro{NPM} {Gerenciador de pacotes do Node.js, do inglês \textit{Node Package Manager}}
\acro{PHP} {\textit{PHP: Hypertext Preprocessor}}
\acro{REST} {Transferência de Estado Representativo, do inglês \textit{Representational State Transfer}}
\acro{SSD} {Unidade de estado sólido, do inglês \textit{Solid state drives}}
\acro{SQL} {Linguagem de Consulta Estruturada, do inglês \textit{Structured Query Language}}
\acro{URI} {Identificador de recurso uniforme, do inglês \textit{Uniform Resource Identifier}}
\acro{URL} {Localizador uniformizado de recursos, do inglês \textit{Uniform Resource Locator}}
\acro{VPS} {Servidor privado virtual, do inglês \textit{Virtual Private Server}}



\acro{BER} {\textit{Bit Error Rate}}
\acro{BT} {\textit{Block Tri-diagonal}}
\acro{CG} {\textit{Conjugate Gradient}}
\acro{CSS} {\textit{Chirp Spread Spectrum}}
\acro{DCC-MAC} {\textit{Dynamic Channel Coding} MAC}
\acro{DSSS} {\textit{Direct-sequence spread spectrum}}
\acro{EP} {\textit{Embarrassingly Parallel}}
\acro{FIFO} {\textit{First In, First Out}}
\acro{FT} {\textit{Fast Fourier Transform}}
\acro{IEEE} {\textit{Institute of Electrical and Electronics Engineers}}
\acro{IR-UWB} {\textit{Impulse Radio} UWB}
\acro{IS} {\textit{Integer Sort}}
\acro{LU} {\textit{Lower-Upper Gauss-Seidel}}
\acro{MAC} {\textit{Medium Access Control}}
\acro{MB-OFDM} {\textit{Multi-Band Orthogonal Frequency Division Multiplexing}}
\acro{MG} {\textit{Multi-Grid}}
\acro{MPE} {MPI \textit{Parallel Environment}}
\acro{MPI} {Interface de passagem de mensagem, do inglês \textit{Message Passing Interface}}
\acro{NAS} {NASA \textit{Advanced Supercomputing}}
\acro{NOAH} {\textit{No Ad-Hoc Routing Agent}} 
\acro{NS-2} {\textit{Network Simulator} 2}
\acro{NoCs} {Redes-em-Chip, do inglês \textit{Networks-on-Chip}}
\acro{NPB} {NAS \textit{Parallel Benchmark}}
\acro{OpenMP} {Multi-processamento aberto, do inglês \textit{Open Multi-Processing}}
\acro{PHY} {\textit{Physical Layer}}
\acro{PPM} {\textit{Pulse-Position Modulation}}
\acro{SP} {\textit{Scalar Penta-diagonal}}
\acro{TH} {\textit{Time Hopping}}
\acro{THS} {\textit{Time Hopping Sequence}}
\acro{UDP}{\textit{User Datagram Protocol}}
\acro{UWB} {\textit{Ultra Wide Band}}
\acro{WiNoCs} {Redes-em-Chip Sem Fio, do inglês \textit{Wireless Networks-on-Chip}}
\acro{WLAN} {Rede Local Sem Fio, do inglês \textit{Wireless Local Area Network}}
\acro{WPAN} {Rede de Área Pessoal Sem Fio, do inglês \textit{Wireless Personal Area Network}}
\end{acronym}

\makeatletter
\renewcommand\numberline[1]{#1\hspace{0.8em}}
\makeatother

% Altera para espaçamento simples a partir daqui
\singlespacing

% Sumário
\tableofcontents

% Altera para espaçamento 1,5 a partir daqui
\onehalfspacing

%% TEXTUAIS 
% Para forçar que elementos textuais e pós-textuais sejam impressos no anverso e verso das folhas
\setboolean{@twoside}{true}
% Altere o número da página para o correto. Conte todas as páginas frente e verso, menos a capa, nclusive a ficha catalográfica até a página do primeiro capítulo.
\setcounter{page}{25}

\renewcommand{\thefootnote}{\arabic{footnote}}

% Capítulos
% Para forçar que o capítulo de introdução comece no anverso
\setboolean{@openright}{true}
% Nome do capítulo
\chapter{Introdução}
% Label para referenciar
\label{introducao}

% Diminuir espaçamento entre título e texto
\vspace{-1.9cm}

% Texto do capítulo
  
  A internet como mídia de comunicação, possui o mais amplo de todos os alcances, 
  conforme pode ser visto na pesquisa do IBGE de 2005 para 2011, 
  o número de internautas cresceu 143.8\%  e o de pessoas com celular, 107.2\%. 
  E para continuar a oferecer serviços e informações, com rapidez e até mesmo em tempo real, 
  é necessário se preocupar com a quantidade de milhões de usuários simultâneos, 
  que cresce exponencialmente, e vencer barreiras tecnológicas de escalabilidade e desempenho nos servidores.
  
  Em sistemas web desenvolvidos sob as plataformas tradicionais como JAVA, \ac{PHP}, .NET dentre outros 
  é necessário paralisar um processamento enquanto utiliza uma entrada e saída do servidor. 
  Essa paralisação é conhecida como um modelo bloqueante. Exemplificando este modelo, em um servidor Web 
  que cada processo é uma requisição de feita pelo usuário. Com o decorrer novos usuários realizam novas 
  requisições aumentando o processamento. No modelo bloqueante cada requisição é enfileirada e depois 
  processadas uma a uma. Enquanto uma requisição esta sendo processada as demais ficam em espera, 
  mantendo-se ociosas por um período indeterminado na fila.\cite{Pereira:2013}
  
  Com esta arquitetura tradicional, gasta-se muito tempo mantendo uma fila de espera com processos ociosos,
  tais como: envio de e-mails, consultas em banco de dados, leitura em disco que não liberam recursos enquanto
  não forem finalizadas. Com o aumento dos acessos ao sistema é necessário fazer uma atualização
  do hardware (equipamento).\cite{Pereira:2013}
  
  \citeonline{Abernethy:2011}, explica que em linguagens como Java e PHP, cada conexão cria-se uma 
  nova \textit{thread} com 2 MB de memória RAM. Se em um servidor possuir 8 GigaBytes de memória RAM, 
  teoricamente o número máximo de conexões concorrentes é aproximadamente 4.000 usuários. 
  Com o aumento da base de cliente, e claro, se quiser que o aplicativo web suporte mais usuários, é necessário 
  adicionar mais e mais servidores. Como foi descrito, o gargalo em toda a arquitetura 
  da aplicação web ( incluindo a velocidade de tráfego, velocidade do processador e velocidade da memória RAM) 
  estaria associado ao número máximo de conexões concorrentes que um servidor pode manipular.
  
  Portanto, observa-se que o escalonamento horizontal, adicionando novos servidores, além do custo altíssimo, 
  torna a arquitetura do sistema complexa pois será necessário acrescentar servidores de balanceamento, 
  rede estruturada da central de dados que seja capaz de suportar um alto tráfego e acompanhamento dos processos 
  do sistema de perto para que os bloqueios sejam consertados em tempo hábil. A utilização de escalonamento vertical, 
  ou melhor, atualização de hardware – colocando mais processadores ou memória - pode inviabilizar a arquitetura do 
  sistema, visto que há uma barreira de hardware, mais especificamente, placas-mãe que não suportam mais de 8 slots 
  de memória ou determinados modelos de memória RAM, suporte a processadores com mais de 7 núcleos. 
  Além dessas limitações tecnológicas, ha o agravante do alto custo para atualizar este hardware. 
  Processadores com 7 núcleos são caros e dependendo dos casos é necessário trocar todo o equipamento - hardware - 
  para garantir o devido funcionamento dos componentes.
  
  Pelos problemas citados acima surgiu a necessidade de resolver este problema, 
  em nível de software, que permita receber um grande número de conexões simultâneas 
  nos servidores, capaz de ser escalável e consumir menores índices de memória RAM.
  
  Para resolver este problema adota-se o paradigma de programação orientada a eventos utilizando
  as caracteriticas nativas do JavaScript: modelo de eventos assíncronos, funções anônimas e \textit{callbacks}. \cite{Junior:2012}
  
  Como \citeonline{Junior:2012} exemplificou, um programa assíncrono ao fazer uma requisição 
  a um banco de dados especifica o que deve ser feito com os resultados do banco de dados, ele não espera a 
  finalização da requisição e continua a processar outras atividades existes no programa. 
  Apenas quando o resultado da requisição é retornado do banco de dados, a codificação para manipular os estes dados 
  é executado. A esta lógica de programação, executada após todos os dados serem retornados, dá-se ao nome de callback.\cite[p. 2]{Junior:2012} 
  
  Com as característica nativas do JavaScript e o problema a ser resolvido criou-se o ambiente de desenvolvimento Node.Js, 
  que é melhor descrito por \citeonline{Junior:2012} como uma plataforma cujo o objetivo é a fácil 
  construção de rápidas e escaláveis aplicações de rede. Para isto o Node.Js emprega orientação 
  a eventos utilizando o JavaScript Engine V8 do Google, operações de entradas e saídas em eventos (assíncronos) 
  e não bloqueantes. \citeonline{Abernethy:2011} cita que ao invés de criar novas \textit{threads}
  no sistema operacional para cada conexão e alocar a memória RAM que acompanha essas \texit{threads}, 
  cada conexão dispara um evento executado no processo do motor Node.Js. 
  
  O Node.Js afirma que ele nunca irá ter bloqueios ou impasses, já que bloqueios não é uma característica 
  da sua plataforma mesmo em processamento de entradas e saídas e que um servidor pode suportar 
  dezenas de milhares de conexões simultâneas.\cite{Abernethy:2011}
  
  Pretende-se com esta proposta de pesquisa investigar e elaborar aplicações para a internet, 
  utilizando o servidor Node.Js e comparar a sua performance em relação ao desenvolvimento
  de aplicativos tradicionais.
  
  
\section{Motivação}
\label{motivacao}
  
  Com alto crescimento da internet e de serviços web em REST tem-se os aplicativos web tem a necessidade de ter
  uma arquitetura, de baixo custo em hardware, escalável e que seja capaz de suportar milhares de usuários 
  sem perdas de dados ou conexões. 
  
\section{Objetivos}
\label{objetivos}

\subsection{Objetivos Geral}

  Pretende-se com esta pesquisa investigar, comparar e demonstrar a capacidade 
  do ambiente Node.Js de processar e responder milhares de requisições comparando-o com um ambiente Python.
  
  Para isso serão utilizados dois aplicativos desenvolvidos como uma API RESTFul provendo as operações
  básicas como \ac{CRUD} de uma lista de contatos.
  
\subsection{Objetivos específicos}

  Os objetivos específicos deste trabalho são:
  
    \begin{compactitem}
      \item[a)] Estudo teórico do ambiente de desenvolvimento Node.Js para construção do protótipo 
      \item[b)] Propor dois protótipos e comparár a sua forma de desenvolvimento.
      \item[b)] Realizar testes de carga nos protótipos para verificar a perfomance com um alto número de conexões,
      seu comportamento em relação ao consumo de hardware e apresentá-los ao leitor.
    \end{compactitem}
  
  
\section{Organização}
\label{organizacao}  

  Para contextualizar o leitor, o Capítulo \ref{ambiente-node-js} aborda o referêncial teórico e fontes de estudos utilizados para iniciar com o ambiente 
  Node.Js. Já o Capítulo \ref{desenvolvimento-prototipos} compreende a lista de requisitos e especificações do protótipo a ser construído e o
  desenvolvimento. O Capítulo \ref{experimentos-resultados} descreve os testes realizados e os resultados obtidos. 
  Por fim, o Capítulo \ref{}, conclui o trabalho acadêmico.
% Os demais capítulos não precisam começar no anverso
\setboolean{@openright}{false}
% Nome do capítulo
\chapter{Metodologia}
% Label para referenciar
\label{metodologia}

% Diminuir espaçamento entre título e texto
\vspace{-1.9cm}

% Texto do capítulo
  Baseando-se em metodologias de desenvolvimento de software ágil Scrum, foram determinadas 
  as seguintes etapas para elaboração deste trabalho:


\section{Estudo do ambiente Node.Js}
  
  \begin{compactitem}
    \item[a)] Estudar o paradigma de orientação a eventos.
    \item[b)] Estudar o Event Loop.
    
    Busca um entendimento básico sobre o ciclo de eventos (Event-Loop) e como ele é utilizado no NodeJs.
    
    \item[c)] Estudar as principais características do Node.Js.
    
    Conhecer o modelo single-thread, diferenças entre o modelo assíncrono e síncrono, dentre outros.
    
    \item[d)] Estudo do framework Express
    
    Conhecer o framework feito em Node.Js que será a base para criar a 
    aplicação devido a grandes módulos já inclusos e se codificar de maneira \ac{REST}.

  \end{compactitem}
  
\section{Levantamento de requisitos}

  \begin{compactitem}
    \item[a)] Estudar a arquitetura web
    
    Buscando um entendimento básico de como é a implementação de uma aplicação 
    \ac{REST} e seus protocolos relacionados ao HTTP.
    
    \item[b)] Especificar requisitos da aplicação.
    
    Criar um modelo de \ac{API} da aplicação para fácil entendimento e
    independente da plataforma de desenvolvimento.
    
    \item[c)] Especificar serviços da aplicação.
    
    Detalhar e documentar o serviço de hospedagem na internet, além de outros componentes 
    necessários para o funcionamento do ecossistema da aplicação como:
    
      \begin{compactitem}
	\item[-] Hardware utilizado;
	\item[-] Servidor web para responder requisições na porta 80;
	\item[-] Serviços adicionais instalados ( Banco de dados );
	\item[-] Softwares para monitorar desempenho e utilização do servidor;
      \end{compactitem}
      
  \end{compactitem}

\section{Criar dois aplicativos em diferentes paradigmas}
  
  \begin{compactitem}
    \item[a)] Desenvolvimento da aplicação no paradigma orientado a eventos.
    
    Desenvolver um aplicativo escrito no ambiente Node.Js com o framework Express.Js
    
    \item[b)] Desenvolvimento do mesmo aplicativo em outros ambientes.
    
    Busca-se com esta etapa ter um aplicativo escrito na linguagem Python com o framework Django
    para comparamos o desempenho, do ambiente aqui estudado.
    
  \end{compactitem}

\section{Especificar testes e comparar resultados}

  \begin{compactitem}
    \item[a)] Utilizar testes de carga na \ac{API}.
    
    Realizar um plano de testes de estresse e carga nas \ac{API} desenvolvidas através
    da ferramenta loader.io \footnote{\label{noteloader}http://loader.io}, melhor descrita na Sub-Seção \ref{ferramentas-utilizadas-para-testes}.
    
    
    \item[b)] Descrever o software de testes.
    
    Descrever passo a passo como gerar um teste de carga na aplicação dos serviços descritos acima.
    
    \item[c)] Executar os testes para avaliar o desempenho das aplicações.
    
    Executar os testes para obter informações e medir o desempenho de cada aplicação.
    
    \item[d)] Avaliar os resultados obtidos após a analise, coleta e definições das métricas.
    
    Por último realizar uma análise sobre o desempenho positivo ou negativo de cada modelo desenvolvido.
    
%    Definir aqui as métricas passadas pelo serviço web loader.io.
    
  \end{compactitem}

% testes de carga: Testes realizados para verificar se um sistema suporta uma determinadas
% carga. Volume do trafego para um determinado sistema. Geralmetne medida em transacoes
% requisicoes dos ususarios

% Transação: operação completa no sistema. Por exemplo, buscar um produto.
% Sistema: todo o conjunto de servidores, rede entre servidores, softwares de terceiros e a aplicação.
% Utilização de um recurso: percentual, em uso, do total de recursos disponíveis.
% Tempo de resposta: Tempo desde o momento em que o usuário envia a requisição até o momento em que recebe a resposta completa.
% Na nomeclatura do Jmeter , é o elapsed time

% Profiling: instrumentação da aplicação para estudo dos métodos e seus tempos de execução.
% Vazão: taxa com que um sistema responde às requisições recebidas.
% Gargalo: tudo o que impede que o sistema apresente maior vazão.
% Se a vazão for inferior à taxa com que as requisições são enviadas ao sistema.

% Monitoramento do sistema
% Métricas Sistema Operacional (Todas as máquinas)
% Utilização CPU, memória, Swap, disco, rede, etc.
% /proc, perfmon, vmstat, etc.

% Métricas Banco de Dados
% Tempo ocupado, tempo execução médio por query, locks, leituras físicas e lógicas, etc.
% Relatório AWR, etc.

% Profiling(Servidor de aplicação)
% Tempo de execução por método, tempo de execução em CPU por método, memória consumida por classe, etc.
% VisualVM, RedGate Ants, etc.

% NewRelic
% Integra profiling e monitoramento de métricas do sistema operacional.

% testes de estresse: Testes realizados para determinar
% a capacidade maxima do sistema.

\subsection{Ferramentas Utilizadas}
\label{ferramentas-utilizadas-para-testes}
  
  Para realizar os testes de carga e estresse será utilizado o serviço em computação nas nuvens
  da empresa SendGrid\footnote{http://labs.sendgrid.com/} denominado loader.io.
  
  De acordo com o sitio o serviço é definido com um serviço de teste de carga livre,
  que permite realizar testes de estresse em aplicativos web ou \ac{API} com milhares de conexões simultâneas
  (tradução nossa).
  
  Na documentação\footnote{http://support.loader.io/} do serviço, mais especificamente, na seção começando (tradução nossa)
  temos as seções de como criar o teste, tipos de teste, verificando um aplicativo, variáveis,
  resultados de testes(tradução nossa). Neste capítulo não iremos abordar todos os items pois serão melhor explicados
  no Capítulo \ref{}. Apenas a seção tipos de testes será explicada neste momento.
  
  Os três tipos de testes suportados pelo serviço são:
  
  Clientes por testes
  
  Este teste permite que especifique um número total de clientes que se conectam ao serviço. Quando criar o teste,
  especifique somente um número de clientes então vários clientes irão se conectar ao longo da duração do teste. 
  Por exemplo, se for criado um teste com 20.000 clientes dentro de 20 segundos, o serviço irá executar a carga de 
  1.000 clientes por segundo. (tradução nossa)
  
  Clientes por segundo
  
  Este teste permite que especifique um número de clientes que se conectam a cada segundo. Por exemplo se for criado
  um teste com 1.000 clientes dentro de 20 segundos, o serviço irá conectar 20.000 clientes por teste.
  
  Mantendo carga do clientes
  
  Segundo a definição da documentação, este teste é utilizado para sobrecarregar (martelar) o site ou \ac{API}.
  
  O serviço loader.io garante que um número constante de clientes estará consumindo e realizando requisições em 
  sua \ac{API} a todo o momento.
  
  Este teste permite que especifique um número mínimo e máximo de clientes. Especificando zero clientes
  até 10.000, por exemplo, o teste vai começar com zero até 10.000 clientes simultâneos no final do teste.
  



% PÓS-TEXTUAIS %%
% Bibliografia no arquivo 'Dissertacao.bib'
% Alterar o título das referências para somente 'Referências'
\renewcommand{\bibname}{Referências}
%\bibliographystyle{abnt-alf}
%\bibliography{Dissertacao}

% Para forçar que os apêndices e anexos comecem no anverso
%\setboolean{@openright}{true}

\apendice
\begin{apendice}
%------------------------------------------------------------------------------------------------------------------------------------------------------
% Reiniciar numeração das figuras que aparecem no apêndice
\setcounter{figure}{0}

\chapter{Primeiro apêndice}
\label{apend:express-skel}

% Para diminuir espaçamento entre o título e o texto
\vspace{-1.9cm}

\lstinputlisting[
  language=c, 
  label=leitura-arquivos-diretorio-node, 
  caption={Fonte: \cite{pereira}. O inferno em chamadas de retorno}
]{pos-texto/leitura-arquivos-diretorio-node.js}

\lstinputlisting[
  language=c, 
  label=leitura-arquivos-diretorio-node-callback-heaven, 
  caption={Fonte: \cite{pereira}. Chamadas de retorno organizada}
]{pos-texto/leitura-arquivos-diretorio-node-callback-heaven.js}

\lstinputlisting[
  language=c, 
  label=leitura-arquivos-diretorio-node-strongloop, 
  caption={Fonte: \cite{strongloop}. Exemplo complexo de leitura de arquivos com callbacks hell}
]{pos-texto/leitura-arquivos-diretorio-node-strongloop.js}

\lstinputlisting[
  language=c, 
  label=leitura-arquivos-diretorio-node-strongloop-modular, 
  caption={Fonte: \cite{strongloop}. Exemplo complexo de leitura de arquivos modularizado}
]{pos-texto/leitura-arquivos-diretorio-node-strongloop-modular.js}

\lstinputlisting[
  language=c, 
  label=node-js-app, 
  caption={Fonte: Autor. Arquivo central para aplicação Node}
]{pos-texto/app.js}

\lstinputlisting[
  language=c,
  label=node-js-contact-route, 
  caption={Fonte: Autor. Módulo para a API de contato}
]{pos-texto/contacts.js}
\end{apendice}

\anexo
% Nome do Anexo
\chapter{Primeiro Anexo}
\label{primeiro-anexo}
% Para diminuir espa�amento entre o t�tulo e o texto
\vspace{-1.9cm}

% Texto
% Textos ou documentos não elaborados pelo autor.

\lstinputlisting[
  language=c, 
  label=leitura-arquivos-diretorio-node, 
  caption={Fonte: \cite{Pereira:2013}. O inferno em chamadas de retorno}
]{pos-texto/leitura-arquivos-diretorio-node.js}

\lstinputlisting[
  language=c, 
  label=leitura-arquivos-diretorio-node-callback-heaven, 
  caption={Fonte: \cite{Pereira:2013}. Chamadas de retorno organizada}
]{pos-texto/leitura-arquivos-diretorio-node-callback-heaven.js}

\lstinputlisting[
  language=c, 
  label=leitura-arquivos-diretorio-node-strongloop, 
  caption={Fonte: \cite{Strongloop:2013}. Exemplo complexo de leitura de arquivos com callbacks hell}
]{pos-texto/leitura-arquivos-diretorio-node-strongloop.js}

\lstinputlisting[
  language=c, 
  label=leitura-arquivos-diretorio-node-strongloop-modular, 
  caption={Fonte: \cite{Strongloop:2013}. Exemplo complexo de leitura de arquivos modularizado}
]{pos-texto/leitura-arquivos-diretorio-node-strongloop-modular.js}


\end{document}