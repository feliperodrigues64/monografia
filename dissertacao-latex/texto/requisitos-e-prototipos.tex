
\chapter{Desenvolvimento dos protótipos}
% Label para referenciar
\label{desenvolvimento-prototipos}

% Diminuir espaçamento entre título e texto
\vspace{-1.9cm}

% Texto do capítulo

  Este capítulo apresenta o cenário proposto para o estudo de caso, bem
  como o desenvolvimento dos protótipos com suas fases de implementação, iniciando
  pela elicitação dos requisitos seguindo a abordagem \ac{REST}.
  
  Como serão desenvolvidos dois protótipos com o objetivo de compará-los, iniciaremos o desenvolvimento com o \textit{framework} Django, e por fim, faremos o desenvolvimento
  com o ambiente Node.Js e \textit{framework} Express.Js, formando assim a aplicação proposta.
  
\section{O escopo do projeto}
\label{escopo-projeto}

  Com base nos aspectos abordados ao longo deste trabalho, 
  esta seção visa apresentar o desenvolvimento de um \textit{Web Service} seguindo os padrões \ac{REST} e a utilização 
  de um banco de dados relacional Postgres para a persistência dos dados.  
  Tal serviço possui um método que será consumido pelos dispositivos clientes.

\subsection{Levantamento de Requisitos}
\label{levantamento-requisitos}

  Utilizaremos o mesmo exemplo citado por \citeonline{pereira} pois é de fácil compreensão
  e entendimento de uma simples aplicação. Em seu livro, o autor, para ensinar o \textit{framework}
  Express.Js e Node.Js, cria uma agenda de contatos integrando-o com um Web \textit{chat} funcionando em tempo real.
  
  A aplicação do autor possui requisitos funcionais bastante objetivos, tais como: O usuário cria, edita ou exclui um contato; 
  o usuário realiza \textit{login} informando nome e e-mail; conectar ou desconectar no bate-papo; poder enviar e receber mensagens 
  no bate-papo somente dos contatos \textit{online}.
  
  Ao investigar o primeiro requisito (criar, editar, atualizar ou excluir um contato) foi possível encaixá-lo neste trabalho,
  pois um serviço em REST resolverá a necessidade de buscar os dados através do protocolo HTTP 
  podendo ser consumido em qualquer dispositivo cliente. Como este trabalho não possui o mesmo foco do aplicativo utilizado por \citeonline{pereira}
  e sim comparar desempenho das aplicações, simplificamos o desenvolvimento retirando as funcionalidades
  de \textit{login} e qualquer outra funcionalidade correspondente ao bate-papo. Apenas as operações de criar, editar,
  atualizar ou excluir de contatos foi mantida e a adição de pessoas (criar, editar, atualizar ou excluir uma pessoa)
  relacionada a um ou vários contatos para cumprir o objetivo deste trabalho.

  Sendo assim, é possível destacar os seguintes requisitos candidatos:

  \begin{compactitem}
    \item[a)] A \ac{API} \ac{REST} deverá ter um recurso chamado Pessoas.
    \item[b)] A \ac{API} \ac{REST} deverá prover estratégias para manipular as ações de CRUD de uma pessoa(s)
    \item[c)] A \ac{API} \ac{REST} deverá ter um recurso chamado Contatos.
    \item[d)] A \ac{API} \ac{REST} deverá prover estratégias para manipular as ações de CRUD de um contato(s)
  \end{compactitem}
  
\subsubsection{Análise dos requisitos}
  
  Perante os requisitos elicitados verificou-se que o objetivo essencial de uma pessoa, candidato a usuário
  do sistema, é controlar os próprios contatos.
  O principal requisito destacado na análise é registrar estes contatos. Para atender o requisito
  propõem-se a utilização de uma \ac{API} \ac{REST}, para fornecer os \ac{CRUD} de cada recurso.
  
  Para os testes será utilizado apenas a operação de buscar os contatos utilizando o método GET 
  do protocolo HTTP, pois a maioria de dispositivos consumidores consomem esse tipo de
  recurso com maior frequência.
 
  \begin{table}[H]
    \centering
    \footnotesize
    \vspace{0.5cm}
    % Alterar espaçamentos antes e depois do caption
    \setlength{\abovecaptionskip}{0pt}
    \setlength{\belowcaptionskip}{0pt}
    % Caption
    \caption[Descrição da API de contatos]{Descrição da API de contatos}
    \label{tab:api-descricao-contato}
    % Conteúdo da tabela
    \begin{tabular}{c|c|c|p{8cm}}
      \hline \hline
      Metódo  &	Parâmetro &	Recurso &	Descrição \\
      \hline \hline
      GET	& -	& contatos	& Retorna a lista de contatos 
					  cadastrados no banco de dados \\
      \hline \hline
    \end{tabular}
    % Fonte
    \captionfont{\small{\textbf{\\Fonte: Autor}}}
  \end{table}
  
\subsubsection{Tecnologias Utilizadas}


  O aplicativo comparativo em Django, utiliza da linguagem de programação Python, o banco de dados relacional Postgres,
  o \textit{framework} Django para construir o aplicativo Web e o pacote Django Rest Framework responsável por adicionar
  funcionalidades de uma API REST ao \textit{framework}
    
  Já o aplicativo Node.JS, utiliza o JavaScript conforme descrito no capítulo \ref{ambiente-node-js}, o banco de dados
  relacional Postgres e o \textit{framework} Express.Js para projetar o aplicativo Web.
 
\subsubsection{Mapeamento de dados}

  De acordo com os requisitos elicitados foi construído o diagrama de entidade e relacionamento para melhor
  documentar estes requisitos.

  Conforme as necessidades das aplicações, os protótipos possuem as seguintes tabelas: \textit{Person} e \textit{Contact},
  que fazem o registro de pessoas e contatos, respectivamente, e possui os atributos (campos) necessários
  para armazenar os registros da API no banco de dados. Na entidade de \textit{person}, temos o identificador
  "id" como chave primária, única e auto-incremento, "created" e "modified" como campos de data e o campo
  "nome" para identificar o nome da pessoa. Na tabela de \textit{contact} temos um identificador "id" único e auto-incremento,
  "created" e "modified" como campos de data, "kind" com caracter para saber o tipo do contato e "value" para 
  saber o valor do tipo do contato. A tabela \textit{contact} possui um relacionamento com a tabela \textit{person},
  com a cardinalidade de muitos para um.

  
\section{Aplicação em Django}
\label{desenvolvimento-django}

  Esta seção apresenta o desenvolvimento de uma aplicação Django com o pacote Django Rest Framework, para que o 
  leitor tenha conhecimento geral sobre o processo de desenvolvimento com este \textit{framework}. 
  
  
\subsection{Instalação Base}

 
  Este protótipo foi desenvolvido no sistema operacional Linux, distribuição Ubuntu 14.04. Nada impede ao desenvolvedor 
  utilizar outros sistemas operacionais como Windows ou MacOs pois o Python como linguagem de programação 
  é multiplataforma. 
  
  Para iniciar o desenvolvimento com o \textit{framework} Django é necessário ter instalado no sistema operacional
  bibliotecas: python-pip; python-virtualenv; python-dev; libpq-dev. O python-pip é utilizado para instalar os 
  pacotes do repositório de pacotes do python conhecido como PyPI \footnote[9]{The Python Package Index. Disponível em https://pypi.python.org/pypi}. 
  O python-virtualenv é utilizado para isolar em um ambiente virtual as bibliotecas instaladas e utilizadas na aplicação 
  para que não corrompa as bibliotecas do sistema operacional. O python-dev é um conjunto de ferramentas para o 
  desenvolvimento com a linguagem python e a libpq-dev é a biblioteca responsável por compilar e comunicar a aplicação 
  com o \textit{backend} do banco de dados Postgres.
  
  Após instalar essas bibliotecas no sistema operacional, seguem-se as etapas abaixo.
  
  \begin{compactitem}
    \item[1)] Criar o ambiente virtual;
    \item[2)] Instalar o \textit{framework} Django e o pacote \textit{Django Rest Framework};
    \item[3)] Criar o esqueleto ou estrutura de diretórios da aplicação.
  \end{compactitem}
  
  Para o item 1, execute o comando \textit{virtualenv django\_rest}. Em seguida é necessário ativar este ambiente com o comando
  \textit{source django\_rest\/bin\/activate}. Após a ativação do ambiente prosseguimos para segunda etapa executando o comando
  \textit{ pip install django djangorestframework}
  
  O Django Rest Framework, como descrito em seu site \footnote[10]{Django REst Framework. Disponível em http://www.django-rest-framework.org/} é uma poderosa
  e flexível caixa de ferramentas para construir API\'{}s em aplicações Web. Dentre suas facilidades pode-se destacar: navegação na API através
  de páginas HTML, políticas de autenticação, serializar os objetos do banco de dados, funções e classes base para prover os recursos
  de uma API e suas respectivas funcionalidades, uma extensa documentação e suporte pela comunidade.
  
  Após a visão geral do pacote, executamos a última etapa para criar a estrutura de diretórios. Para isto use o comando
  \textit{django-admin.py startproject [nome do projeto] } ou conforme utilizado neste protótipo foi utilizado um \textit{template}
  \textit{Django Project Template}\footnote[11]{Repositório de código do template Django. Disponível em https://github.com/lucassimon/django-project-template}.
  
  Com este comando ira será criada toda a estrutura necessária para a aplicação faltando poucas alterações e configurações.
  
\subsection{Configurações}

  Entre no diretório criado e faça a instalação dos demais pacotes necessários com o comando \textit{pip install -r requirements\/dev.txt}.
  Em seguida edite o arquivo \textit{settings.py} localizado dentro do diretório \textit{infra\_confs\/} e altere a configuração
  do nome do banco de dados na linha \textit{DATABASE\_URL}.
  
  Altere o arquivo \textit{base.py} dentro do diretório \textit{django\_rest\/settings\/} inserindo na configuração \textit{INSTALLED\_APPS}
  o texto \textit{'rest\_framework'}. Neste mesmo arquivo é necessário setar uma configuração do pacote Django Rest Framework
  com o objetivo de liberar a permissão de acesso para todas os recursos providos no módulo \textit{api}.
  
% algoritmo
  % \begin{figure}[ht]
  \begin{center}	
    % Arquivo da figura
    % 	\caption[\hspace{0.1cm} Texto da figuras.]{Algorítmo CAC RD Neural}
    \textbf{Algoritmo 1 -  Configurações do Rest Framework}
    \vspace{-0.3cm}
    \begin{minipage}[ht]{13cm}
      \begin{algorithm}[H]
      \footnotesize
      \caption{Configuração do Rest Framework}
      \label{alg:conf_rest_framework}
	\begin{algorithmic}[1]
	  \STATE REST\_FRAMEWORK = { DEFAULT\_PERMISSION\_CLASSES: [ rest\_framework.permissions.AllowAny ] }
	\end{algorithmic}
      \end{algorithm}
      % \vspace{-0.3cm} % espaço entre algoritmo e fonte

      \small \centering \textbf{\footnotesize Fonte: Autor.}
    \end{minipage}
  \end{center}
  % \end{figure}
  
  Para finalizar as configurações executamos o comando \textit{python manage.py syncdb --migrate --settings=django\_rest.settings.dev}
  para sincronizar as configurações com o banco de dados criado.
  
\subsection{Desenvolvimento}

  Abaixo segue a metodologia e processos utilizados no desenvolvimento da API em Django.
  
  Para iniciar o desenvolvimento do protótipo é criado o módulo \textit{api}, como descrito anteriormente,
  para isto execute o comando \textit{python manage.py startapp api}. Neste protótipo foi utilizado um \textit{template}
  \textit{Django App Template}\footnote[12]{Repositório de código do template Django. Disponível em https://github.com/lucassimon/django-app-template}.
  
  Através deste comando é criado um diretório de nome \textit{api} contendo os principais arquivos utilizados no decorrer deste protótipo.

\subsubsection{Modelos}

  Primeiro, é necessário escrever o modelos através das classes \textit{Person} e \textit{Contacts}. Estas classes irão representar o diagrama
  de entidade de acordo com o sub-seção Mapeamento de dados, descrita anteriormente.
  Na classe \textit{Person} foi definido a coluna \textit{name} para salvar o nome da pessoa.  
  Na classe \textit{model} \textit{Contact} temos o atributo \textit{person} como chave estrangeira da classe \textit{Person}, 
  o campo \textit{kind} para salvar o tipo do contato e o campo \textit{value} para salvar o valor correspondente ao tipo de contato.
  
  Após construir os modelos é necessário sincronizar com o banco de dados, mas deverá inserir o módulo \textit{api}, no 
  \textit{INSTALLED\_APPS} do arquivo \textit{base.py}. Ao executar o comando \textit{python manage.py syncdb --migrate --settings=django\_rest.settings.dev}
  as tabelas serão criadas no banco de dados com o sufixo \textit{api\_} e o nome do \textit{model}. Por exemplo \textit{api\_person} e \textit{api\_contact}.
  
\subsubsection{Serializadores}

  Os \textit{serializers} são parte fundamental da API pois eles irão transformar os objetos do modelo em dados \ac{JSON} quando for 
  responder as requisições das URLs.
  Foi criado o arquivo \textit{serializers.py} e no início do arquivo foi importado o módulo \textit{serializers} e os modelos
  criados.
  
  % algoritmo
  % \begin{figure}[ht]
  \begin{center}	
    % Arquivo da figura
    % 	\caption[\hspace{0.1cm} Texto da figuras.]{Algorítmo CAC RD Neural}
    \textbf{Algoritmo 2 -  Importando módulos no arquivo serializers.py}
    \vspace{-0.3cm}
    \begin{minipage}[ht]{13cm}
      \begin{algorithm}[H]
      \footnotesize
      \caption{Importando módulos}
      \label{alg:imports_serializers}
	\begin{algorithmic}[1]
	  \STATE from rest\_framework import serializers 
	  \STATE from .models import Person, Contact
	\end{algorithmic}
      \end{algorithm}
      % \vspace{-0.3cm} % espaço entre algoritmo e fonte

      \small \centering \textbf{\footnotesize Fonte: Autor.}
    \end{minipage}
  \end{center}
  % \end{figure}
  
  Neste arquivo criou-se duas classes PersonSerializer e ContactSerializer. Ambas herdam da classe abstrata serializers.ModelSerializer.
  A classe PersonSerializer conterá a classe Meta para definir os metadados providos pela abstração do módulo ModelSerializer. Esses 
  metadados são: \textit{model} atribuindo o modelo \textit{Person} e o metadado \textit{fields} que é uma lista contendo os campos ou colunas
  do modelo a serem serializados.
  
  Para a classe ContactSerializer os metadados são: \textit{model} atribuindo o modelo Contact e o metadado \textit{fields} com todos os
  campos deste modelo. Nesta classe foi realizado uma customização que originou um novo campo chamado \textit{kind\_display}. Este campo
  tem como objetivo humanizar as opções oferecidas pelo campo \textit{kind} retornando a descrição do tipo cadastrado.

\subsubsection{Visões}

  Após criar os modelos e os serializadores têm se de criar os \textit{endpoints} ou recursos da API. No inicio
  do arquivo \textit{views.py} é importado os módulos do Django Rest Framework, dentre eles o mais importante
  é o \textit{generic}, os modelos e as classes serializadoras conforme o Algoritmo \ref{alg:imports_rest_framework_core_views}.
  % algoritmo
  % \begin{figure}[ht]
  \begin{center}	
    % Arquivo da figura
    % 	\caption[\hspace{0.1cm} Texto da figuras.]{Algorítmo CAC RD Neural}
    \textbf{Algoritmo 3 -  Importando módulos no arquivo views.py}
    \vspace{-0.3cm}
    \begin{minipage}[ht]{13cm}
      \begin{algorithm}[H]
      \footnotesize
      \caption{Importando módulos}
      \label{alg:imports_rest_framework_core_views}
	\begin{algorithmic}[1]
	  \STATE from rest\_framework.decorators import api\_view
	  \STATE from rest\_framework.response import Response
	  \STATE from rest\_framework.reverse import reverse
	  \STATE from rest\_framework import generics
	  \STATE from .models import Person, Contact 
	  \STATE from .serializers import PersonSerializer, ContactSerializer
	\end{algorithmic}
      \end{algorithm}
      % \vspace{-0.3cm} % espaço entre algoritmo e fonte

      \small \centering \textbf{\footnotesize Fonte: Autor.}
    \end{minipage}
  \end{center}
  % \end{figure}
  
  As classes PersonList e ContactList herdam da classe abstrata \textit{generics.ListCreateApiView} do modulo
  importado \textit{generics} do Django Rest Framework. Essa classe abstrata possui funcionalidades implementadas para 
  listar (metódo GET) e criar (metódo POST) os objetos do banco de dados. Dentro de cada classe é necessário especificar
  nos atributos \textit{model} e \textit{serializer} correspondente a cada classe. Para a classe PersonList sera o atributo
  \textit{model} será atribuído o valor do modelo \textit{Person} e o atributo \textit{serializer} será atribuído a classe PersonSerializer. 
  O mesmo ocorre para a classe ContactList.
  
  Para que a aplicação criada passe a responder as requisições com o método GET por id, PUT por id e DELETE por id é necessário criar
  as classes PersonDetail e ContactDetail. Estas classes irão herdar da classe abstrata RetriveUpdateDestroyView que já possui 
  funcionalidades implementadas para responder essas solicitações do protocolo \ac{HTTP}.

\subsubsection{Rotas}

  Finalizando o desenvolvimento vamos ligar os modelos, \textit{serializers} e \textit{views} à uma rota de URL\'{}s para que o usuário
  possa requisitar os dados. O primeiro parâmetro da função \textit{url()} é o endereço desta rota, podendo
  usar uma expressão regular. O segundo parâmetro é a função ou \textit{view} associada a rota e o terceiro
  parâmetro o \textit{namespace} ou identificador para a rota.
  
  No arquivo \textit{urls.py} crie as seguintes rotas:
  
  \begin{compactitem}
    \item[a)] \textbf{url(r'\^{}/v1/pessoas/', PersonList.as\_view(), name='person-list')}
    
    Rota que aponta para o \textit{endpoint} PersonList e prove a listagem de pessoas com o método GET, como também
    provê o cadastro de uma pessoa com o método POST
    
    \item[b)] \textbf{url(r'\^{}/v1/pessoas/(?P<pk>d+)/',PersonDetail.as\_view(),name='person-detail')}
    
    Rota que aponta para o \textit{endpoint} PersonDetail e prove a listagem de uma determinada pessoa pela sua chave primária
    através do método GET. Com esta mesma rota é possível atualizar uma pessoa com o método PUT e deleta-la através do 
    método DELETE.
        
    \item[c)] \textbf{url(r'\^{}/v1/contatos/',ContactList.as\_view(),name='contact-list')}
    
    Rota que aponta para o \textit{endpoint} ContactList e prove a listagem de contatos com o método GET, como também
    provê o cadastro de um contato com o método POST
    
    \item[d)] \textbf{url(r'\^{}/v1/contatos/(?P<pk>d+)/',ContactDetail.as\_view(),name='contact-detail')}
    
    Essa rota aponta para o \textit{endpoint} ContactDetail e prove a listagem de uma determinado contato pela sua chave primária
    através do método GET. Com esta mesma rota é possível atualizar um contato com o método PUT e deleta-lo através do 
    método DELETE.
        
  \end{compactitem}
  

\subsection{Observações sobre desenvolvimento Django}

   O desenvolvimento para Django é simples e rápido. Com poucas classes e códigos já é possível construir uma API funcional
   através do pacote Django Rest Framework. Todo o código desenvolvido fica organizado graças a obrigatoriedade de endentação
   do Python.

\section{Aplicação em Express.Js}
\label{desenvolvimento-node}

  Neste capítulo é apresentado o desenvolvimento com o \textit{framework} Express.Js realizando comparações com o protótipo
  anterior. 

\subsection{Instalação base}

  Para instalar o Express.js utilizamos o pacote express-generator \footnote[11]{Express Generator. Disponível em http://expressjs.com/starter/generator.html}.E
  Para instalá-lo é necessário executar o comando \textit{npm install -g express-generator} e depois o comando \textit{express rest-node}.
  Com este comando toda a estrutura de diretórios para uma aplicação será criada. Em comparação com o protótipo Django a instalação é
  simples sem a necessidade de instalar vários pacotes no sistema operacional e ter que isolar o ambiente de desenvolvimento.
  
  O Node.Js possui o arquivo \textit{package.json} responsável por gerenciar dependências de pacotes que serão instalados no projeto 
  e definir o nome do aplicativo, \textit{script} de execução e outras configurações. Neste arquivo foi adicionado o módulo PG \footnote[12]{Módulo PG para conectar ao Postgres. Disponível em https://www.npmjs.org/package/pg}
  para comunicar com o banco de dados Postgres. Em Django esse gerenciamento de pacotes fica localizado no arquivo
  \textit{requirements.txt}. Esta simplificação é um ponto forte que o Node.Js oferece ao desenvolvedor.
  
  Para instalar os pacotes da aplicação é necessário executar o comando \textit{npm install} assim como o Django utiliza o 
  \textit{pip install}.

\subsection{Configuração}

  Ao estudar a estrutura da aplicação pode constatar que o arquivo \textit{app.js} (verificar anexo \ref{primeiro-anexo}, código \ref{node-js-app})
  é o ponto central da aplicação, responsável por importar o \textit{framework} Express.Js e 
  pacotes extras além de realizar as configurações do \textit{framework} com os métodos chamados app.set() e app.use().
  Neste mesmo arquivo, configura-se os ambientes de desenvolvimento, tratamento de erros 404 e 500 do protocolo HTTP. 
  Este arquivo assemelha-se ao arquivo de configuração principal do Django \textit{settings/base.py} do nosso protótipo.

\subsection{Desenvolvimento}
  
  A estrutura inicial provê somente rotas e \textit{templates} HTML para que o desenvolvedor já tenha uma aplicação funcional.
  Porém o Express.js oferece ao desenvolvedor a liberdade de usar qualquer padrão de desenvolvimento como o MVC, MVR e 
  outros. Ou seja, o desenvolvedor não fica preso a estrutura inicial podendo altera-la da forma que desejar. 
  
  Neste protótipo foi utilizado a estrutura inicial como modelo de desenvolvimento utilizando o módulo Router() para que desenvolvedor
  crie explicitamente a rota a ser utilizada e o método pertencente a ela, tais como, \textit{get}, \textit{post}, \textit{put}, \textit{delete} e suas respectivas
  chamadas de retorno. As chamadas de retorno destes métodos possuem dois parâmetros obrigatórios, que são o \textit{req} e \textit{res}. O parâmetro
  \textit{req} possui os dados da requisição feita pelo usuário e o \textit{res} possui os métodos necessários para responder a requisição.
  
  No protótipo Django o seu desenvolvimento pode ser considerado metódico, padronizado e ordenado, pois cada arquivo de um módulo
  deve cumprir o seu papel.

\subsubsection{O módulo contact.js }

  Ao conhecer as configurações do arquivo \textit{app.js} e do Router() foi implementado o módulo contact.js (verificar anexo \ref{primeiro-anexo}, código \ref{node-js-contact-route}) com o objetivo
  de criar os recursos da API e seus requisitos.
  
  Primeiro criou-se o arquivo contact.js, vazio, dentro da pasta routes. No arquivo app.js é necessário importar este módulo criado com a
  instrução abaixo.
  
  \begin{verbatim}
    contacts = require('./routes/contacts')
  \end{verbatim}
  
  Semelhante ao \textit{from package import modulo} utilizado no protótipo Django. Depois 
  registra-se no Express o uso deste módulo para a url '/api' com a instrução \textit{app.use('/api', contacts);}.
  
  Agora é possível escrever a lógica e as regras de negócio no arquivo contact.js. No inicio do arquivo importa o \textit{framework}
  Express.js, o \textit{router()}, o pacote PG e uma variável chamada \textit{conString} que contem a configuração de conexão ao banco de dados.
  
  Para o recurso Pessoa foi construído a rota para url '/v1/pessoas' e atribuído a elas os métodos \textit{get()} e \textit{post()}. Outra rota 
  '/v1/pessoas/:id' foi criada para os métodos \textit{get()}, \textit{put()} e \textit{delete()} de uma pessoa em especifico. Para o recurso \textit{Contact} foi 
  construído a rota para url '/v1/contatos' e atribuído a elas os métodos \textit{get()} e \textit{post()}. Outra rota 
  '/v1/contatos/:id' foi criada para os métodos \textit{get()}, \textit{put()} e \textit{delete()} de um contato em especifico.
  
  Todos os métodos do \textit{router()} seguem a mesma lógica para buscar ou salvar os dados. Para o leitor sera exemplificado o desenvolvimento
  do método \textit{get()} que valerá para os outros métodos. A única diferença será as consultas SQL utilizadas. 
  
  Por exemplo, na chamada de retorno do método \textit{get()} implementa-se a lógica para buscar os dados do banco de dados. Para isso executamos 
  o módulo PG com o método \textit{connect()}. Este método precisa passar no primeiro parâmetro uma \textit{string} contendo informações de conexão
  ao banco de dados, variável \textit{conString} e uma chamada de retorno anônima contendo os parâmetros error, \textit{client} e \textit{done}.
  
  Como descrito no capítulo \ref{ambiente-node-js} a variável de erro trata-se do erro de conexão do método \textit{connect} e deve ser 
  tratada quando existir. Para isso foi inserido uma condição. Se verdadeiro envie o status 400 do protocolo HTTP com o 
  \textit{traceback} do erro; senão existir o erro de conexão utiliza-se o parâmetro \textit{client} executando o método \textit{query()}. Essa função \textit{query()}
  tem como primeiro parâmetro o comando SQL conforme este exemplo: \'{}select * from api\_contact\'{}, para os métodos \textit{post()} utiliza o comando
  'insert', para \textit{put()} 'update' e \textit{delete()} 'delete'. O segundo parâmetro do método \textit{query()} é uma chamada de retorno tendo uma 
  variável de erro e resultados.
  
  Novamente dentro desta chamada de retorno é necessário tratar o erro enviando uma resposta com status 400 e o \textit{traceback} do erro.
  Senão existir o erro na consulta é chamado o método \textit{res.json()} passando um objeto JavaScript que contem os resultados retornado
  pela consulta.

\subsection{Observações sobre desenvolvimento Express.Js}

  Em comparação o Django possui o objeto de mapeamento relacional que facilita o retorno e consultas dos dados sem a necessidade
  de escrever as consultas SQL. O módulo PG utilizado neste protótipo realiza somente a conexão com o \textit{backend} para comunicar-se com o 
  banco de dados, não oferecendo o mapeamento de objetos. Após o fim do desenvolvimento de todo o aplicativo foi encontrado o
  pacote Sequelize \footnote[13]{Biblioteca para mapear objetos relacionais. Disponível em http://sequelizejs.com/} que tem como 
  foco mapear os objetos através de um modelo.
  
  Também foi possível observar que este código gerou um encadeamento de chamadas de retorno em 3 níveis. Conforme o capítulo \ref{ambiente-node-js}
  este é o limite tolerável para que não ocorra o \textit{callback hell}. Uma facilidade encontrada no \textit{framework} é a forma como responde
  os objetos em JSON. Apenas chamado o método \textit{res.json()} conseguimos realizar essa resposta, sendo que não ha necessidade de serializar
  os objetos como no protótipo Django pois toda o ambiente Node.Js é em JavaScript tornando-o homogêneo.