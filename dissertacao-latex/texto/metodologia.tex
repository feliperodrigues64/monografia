% Nome do capítulo
\chapter{Metodologia}
% Label para referenciar
\label{metodologia}

% Diminuir espaçamento entre título e texto
\vspace{-1.9cm}

% Texto do capítulo
  Baseando-se em metodologias de desenvolvimento de software ágil Scrum, foram determinadas 
  as seguintes etapas para elaboração deste trabalho:


\section{Estudo do ambiente Node.Js}
  
  \begin{compactitem}
    \item[a)] Estudar o paradigma de orientação a eventos.
    \item[b)] Estudar o Event Loop.
    
    Busca um entendimento básico sobre o ciclo de eventos (Event-Loop) e como ele é utilizado no NodeJs.
    
    \item[c)] Estudar as principais características do Node.Js.
    
    Conhecer o modelo single-thread, diferenças entre o modelo assíncrono e síncrono, dentre outros.
    
    \item[d)] Estudo do framework Express
    
    Conhecer o framework feito em Node.Js que será a base para criar a 
    aplicação devido a grandes módulos já inclusos e se codificar de maneira \ac{REST}.

  \end{compactitem}

\section{Criar dois aplicativos em diferentes paradigmas}
  
  \begin{compactitem}
    \item[a)] Desenvolvimento da aplicação no paradigma orientado a eventos.
    
    Desenvolver um aplicativo escrito no ambiente Node.Js com o framework Express.Js
    
    \item[b)] Desenvolvimento do mesmo aplicativo em outros ambientes.
    
    Busca-se com esta etapa ter um aplicativo escrito na linguagem Python com o framework Django
    para comparamos o desempenho, do ambiente aqui estudado.
    
  \end{compactitem}

\section{Levantamento de requisitos}

   \begin{compactitem}
      \item[a)] Estudar a arquitetura web
      
      Buscando um entendimento básico de como é a implementação de uma aplicação 
      \ac{REST} e seus protocolos relacionados ao HTTP.
      
      \item[b)] Especificar requisitos da aplicação.
      
      Criar um modelo de \ac{API} da aplicação para fácil entendimento e
      independente da plataforma de desenvolvimento.
      
      \item[c)] Especificar serviços da aplicação.
      
      Detalhar e documentar o serviço de hospedagem na internet, além de outros componentes 
      necessários para o funcionamento do ecossistema da aplicação como:
      
	\begin{compactitem}
	  \item[-] Hardware utilizado;
	  \item[-] Servidor web para responder requisições na porta 80;
	  \item[-] Serviços adicionais instalados ( Banco de dados );
	  \item[-] Softwares para monitorar desempenho e utilização do servidor;
	\end{compactitem}
	
    \end{compactitem}


\section{Especificar testes e comparar resultados}

  \begin{compactitem}
    \item[a)] Utilizar testes de carga na \ac{API}.
    
    Realizar um plano de testes de estresse e carga nas \ac{API} desenvolvidas através
    da ferramenta loader.io \footnote{\label{noteloader}http://loader.io}, melhor descrita na Sub-Seção \ref{ferramentas-utilizadas-para-testes}.
    
    
    \item[b)] Descrever o software de testes.
    
    Descrever passo a passo como gerar um teste de carga na aplicação dos serviços descritos acima.
    
    \item[c)] Executar os testes para avaliar o desempenho das aplicações.
    
    Executar os testes para obter informações e medir o desempenho de cada aplicação.
    
    \item[d)] Avaliar os resultados obtidos após a analise, coleta e definições das métricas.
    
    Por último realizar uma análise sobre o desempenho positivo ou negativo de cada modelo desenvolvido.
    
%    Definir aqui as métricas passadas pelo serviço web loader.io.
    
  \end{compactitem}

% testes de carga: Testes realizados para verificar se um sistema suporta uma determinadas
% carga. Volume do trafego para um determinado sistema. Geralmetne medida em transacoes
% requisicoes dos ususarios

% Transação: operação completa no sistema. Por exemplo, buscar um produto.
% Sistema: todo o conjunto de servidores, rede entre servidores, softwares de terceiros e a aplicação.
% Utilização de um recurso: percentual, em uso, do total de recursos disponíveis.
% Tempo de resposta: Tempo desde o momento em que o usuário envia a requisição até o momento em que recebe a resposta completa.
% Na nomeclatura do Jmeter , é o elapsed time

% Profiling: instrumentação da aplicação para estudo dos métodos e seus tempos de execução.
% Vazão: taxa com que um sistema responde às requisições recebidas.
% Gargalo: tudo o que impede que o sistema apresente maior vazão.
% Se a vazão for inferior à taxa com que as requisições são enviadas ao sistema.

% Monitoramento do sistema
% Métricas Sistema Operacional (Todas as máquinas)
% Utilização CPU, memória, Swap, disco, rede, etc.
% /proc, perfmon, vmstat, etc.

% Métricas Banco de Dados
% Tempo ocupado, tempo execução médio por query, locks, leituras físicas e lógicas, etc.
% Relatório AWR, etc.

% Profiling(Servidor de aplicação)
% Tempo de execução por método, tempo de execução em CPU por método, memória consumida por classe, etc.
% VisualVM, RedGate Ants, etc.

% NewRelic
% Integra profiling e monitoramento de métricas do sistema operacional.

% testes de estresse: Testes realizados para determinar
% a capacidade maxima do sistema.

\subsection{Ferramentas Utilizadas}
\label{ferramentas-utilizadas-para-testes}
  
  Para realizar os testes de carga e estresse será utilizado o serviço em computação nas nuvens
  da empresa SendGrid\footnote{http://labs.sendgrid.com/} denominado loader.io.
  
  De acordo com o sitio o serviço é definido com um serviço de teste de carga livre,
  que permite realizar testes de estresse em aplicativos web ou \ac{API} com milhares de conexões simultâneas
  (tradução nossa).
  
  Na documentação\footnote{http://support.loader.io/} do serviço, mais especificamente, na seção começando (tradução nossa)
  temos as seções de como criar o teste, tipos de teste, verificando um aplicativo, variáveis,
  resultados de testes(tradução nossa). Neste capítulo não iremos abordar todos os items pois serão melhor explicados
  no Capítulo AMBIENTE DE TESTES (colocar a referencia do latex aqui). Apenas a seção tipos de testes será explicada neste 
  momento.
  
  Os três tipos de testes suportados pelo serviço são:
  
  Clientes por testes
  
  Este teste permite que especifique um número total de clientes que se conectam ao serviço. Quando criar o teste,
  especifique somente um número de clientes então vários clientes irão se conectar ao longo da duração do teste. 
  Por exemplo, se for criado um teste com 20.000 clientes dentro de 20 segundos, o serviço irá executar a carga de 
  1.000 clientes por segundo. (tradução nossa)
  
  Clientes por segundo
  
  Este teste permite que especifique um número de clientes que se conectam a cada segundo. Por exemplo se for criado
  um teste com 1.000 clientes dentro de 20 segundos, o serviço irá conectar 20.000 clientes por teste.
  
  Mantendo carga do clientes
  
  Segundo a definição da documentação, este teste é utilizado para sobrecarregar (martelar) o site ou \ac{API}.
  
  O serviço loader.io garante que um número constante de clientes estará consumindo e realizando requisições em 
  sua \ac{API} a todo o momento.
  
  Este teste permite que especifique um número mínimo e máximo de clientes. Especificando zero clientes
  até 10.000, por exemplo, o teste vai começar com zero até 10.000 clientes simultâneos no final do teste.
  
