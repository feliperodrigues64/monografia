% Nome do capítulo
\chapter{Metodologia}
% Label para referenciar
\label{metodologia}

% Diminuir espaçamento entre título e texto
\vspace{-1.9cm}

% Texto do capítulo
  Baseando-se em metodologias de desenvolvimento de software ágil Scrum, foram determinadas 
  as seguintes etapas para elaboração deste trabalho:


\section{Estudo do ambiente Node.Js}
  
  \begin{compactitem}
    \item[a)] Estudar as principais características do Node.Js.
    
    Conhecer o modelo única \textit{thread}, diferenças entre o modelo assíncrono e síncrono, dentre outros.
    
    \item[b)] Estudar o ciclo de eventos do inglês , \textit{Event Loop} e como ele é utilizado no NodeJs.
    
    \item[c)] Estudo do \textit{framework} Express
    
    Conhecer o \textit{framework} feito em Node.Js que será a base para criar a 
    aplicação devido a grandes módulos já inclusos e se codificar de maneira \ac{REST}.

  \end{compactitem}
  
\section{Levantamento de requisitos}

  \begin{compactitem}
    \item[a)] Estudar a arquitetura Web dos protótipos de acordo com o problema proposto.
    
    Buscando um entendimento básico de como é a implementação de uma aplicação 
    \ac{REST} e seus protocolos relacionados ao HTTP.
    
    \item[b)] Especificar requisitos da aplicação.
    
    Criar um modelo de API da aplicação para fácil entendimento e
    independente da plataforma de desenvolvimento.
    
    \item[c)] Especificar serviços da aplicação.
    
    Detalhar e documentar o serviço de hospedagem na Internet, além de outros componentes 
    necessários para o funcionamento do ecossistema.
      
  \end{compactitem}

\section{Criar dois aplicativos em diferentes paradigmas e hospeda-los}
  
  \begin{compactitem}
    \item[a)] Desenvolvimento da aplicação no paradigma orientado a eventos.
    
    Desenvolver um aplicativo escrito no ambiente Node.Js com o \textit{framework} Express.Js
    
    \item[b)] Desenvolvimento do mesmo aplicativo em outro ambiente.
    
    Busca-se com esta etapa ter um aplicativo escrito na linguagem Python com o \textit{framework} Django
    para comparamos o desempenho, do ambiente aqui estudado com o modelo \"tradicional\".
    
    \item[b)] Hospedar os aplicativos em uma VPS.
    
    Hospedar os aplicativos em um servidor provado virtual contendo seu próprio sistema operacional e processos
    especificos para cada aplicação. Para esta necessidade será utilizado os serviços da empresa Digital Ocean \footnote{
    Serviço de hospedagem nas núvens. Disponível em https://www.digitalocean.com/}
    
  \end{compactitem}

\section{Especificar testes e comparar resultados}

  \begin{compactitem}
    \item[a)] Utilizar testes de carga na \ac{API}.
    
    Realizar um plano de testes de estresse e carga nas \ac{API} desenvolvidas através
    da ferramenta loader.io \footnote{Teste de carga baseado em nuvem. Disponível em http://loader.io},
    melhor descrita na Sub-Seção \ref{ferramentas-utilizadas-para-testes}.
    
    
    \item[b)] Descrever o software de testes.
    
    Descrever passo a passo como gerar um teste de carga na aplicação dos serviços descritos acima.
    
    \item[c)] Executar os testes: Clientes por testes; clientes por segundo; manter carga de clientes.
    
    Estes testes possuem a finalidade de medir o tempo de resposta, volume de transações e conexões por segundo. E por fim
    medir a carga com um volume de usuários até a duração de 60 segundos respectivamente. 
    Com os resultados obtidos poderemos avaliar em números o tempo de resposta de cada teste,
    erros de tempo de limite excedido, do inglês \textit{timeout}, erros de rede e conexões suportadas. 
    
    \item[d)] Avaliar os resultados obtidos após a análise, coleta e definições das métricas.
    
    Por último realizar uma análise sobre o desempenho positivo ou negativo de cada modelo desenvolvido.
    
%    Definir aqui as métricas passadas pelo serviço web loader.io.
    
  \end{compactitem}

% testes de carga: Testes realizados para verificar se um sistema suporta uma determinadas
% carga. Volume do trafego para um determinado sistema. Geralmetne medida em transacoes
% requisicoes dos ususarios

% Transação: operação completa no sistema. Por exemplo, buscar um produto.
% Sistema: todo o conjunto de servidores, rede entre servidores, softwares de terceiros e a aplicação.
% Utilização de um recurso: percentual, em uso, do total de recursos disponíveis.
% Tempo de resposta: Tempo desde o momento em que o usuário envia a requisição até o momento em que recebe a resposta completa.
% Na nomeclatura do Jmeter , é o elapsed time

% Profiling: instrumentação da aplicação para estudo dos métodos e seus tempos de execução.
% Vazão: taxa com que um sistema responde às requisições recebidas.
% Gargalo: tudo o que impede que o sistema apresente maior vazão.
% Se a vazão for inferior à taxa com que as requisições são enviadas ao sistema.

% Monitoramento do sistema
% Métricas Sistema Operacional (Todas as máquinas)
% Utilização CPU, memória, Swap, disco, rede, etc.
% /proc, perfmon, vmstat, etc.

% Métricas Banco de Dados
% Tempo ocupado, tempo execução médio por query, locks, leituras físicas e lógicas, etc.
% Relatório AWR, etc.

% Profiling(Servidor de aplicação)
% Tempo de execução por método, tempo de execução em CPU por método, memória consumida por classe, etc.
% VisualVM, RedGate Ants, etc.

% NewRelic
% Integra profiling e monitoramento de métricas do sistema operacional.

% testes de estresse: Testes realizados para determinar
% a capacidade maxima do sistema.

\subsection{Ferramentas Utilizadas}
\label{ferramentas-utilizadas-para-testes}
  
  Para realizar os testes de carga e estresse será utilizado o serviço em computação nas nuvens
  da empresa SendGrid\footnote{Serviços para desenvolvedores. Disponível em http://labs.sendgrid.com/} denominado loader.io.
  
  De acordo com o sítio o serviço é definido como um serviço de teste de carga livre,
  que permite realizar testes de estresse em aplicativos web ou API\'s com milhares de conexões simultâneas.
  
  Este serviço foi escolhido pois oferece testes gratuitos até o máximo 10.000 clientes por teste para duas 
  urls especificas. Sua \textit{interface} para criar os testes é simples, fácil de utilizar e
  configurar os testes. Na execução dos testes o gráfico e os dados são atualizados em tempo real até a duração do teste.
  
  Existe uma documentação no sítio da \citeonline{SupportLoader:2014}, que explica a utilizar a ferramenta para usuários
  iniciantes, principalmente na seção começando. Nesta seção possuias sub-seções de como criar o teste, tipos de teste,
  verificando um aplicativo, variáveis e resultados de testes. Documentos estes essenciais para 
  entender o básico para utilização deste serviço. 
  
  Neste capítulo não irá ser abordado as configurações pois serão melhores explicados
  no Capítulo \ref{experimentos-resultados}.