% Nome do capítulo
\chapter{Considerações finais}
% Label para referenciar
\label{conslusao-e-trabalhos-furutos}

% Diminuir espaçamento entre título e texto
\vspace{-1.9cm}

% Texto do capítulo

  Com o desenvolvimento deste trabalho, ficou claro que a tecnologia Node.Js e suas características
  deve ser utilizada para solucionar o problema de milhares de conexões em servidores web pois consegue 
  cumprir o seu objetivo a qual foi proposto com eficiência.
  
  Um dos pontos a serem considerados, é o desenvolvimento da aplicação pois com o framework Django tem-se um rápido 
  desenvolvimento do software capaz de fornecer uma API robusta. No framework Express.Js o desenvolvimento
  é custoso para desenvolvedores menos experientes pois trata-se de uma nova abordagem e paradigma. 
  O outro ponto a ser considerado é a maturidade do framework Django, ano de 2005, e da linguagem Python. O Node.js
  nasceu em 2009 e possui uma ampla comunidade \textit{open source} que contribui com código e bibliotecas complementares
  ao sistema.
  
  Uma melhoria em nível do protótipo Django seria a utilização de cache de consultas e dos resultados em JSON que poderiam
  reduzir o tempo de resposta e aumentar o número de sucessos. Este cache pode ser colocado no framework e utilizar também
  um software como o varnish em um outro servidor; 

  Após a implementação de todas essas funções, deveria ser feito um novo plano de
  teste para obter dados dos impactos causados pelas modificações efetuadas no cliente e no
  servidor.
