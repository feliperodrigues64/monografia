% Nome do capítulo
\chapter{Introdução}
% Label para referenciar
\label{introducao}

% Diminuir espaçamento entre título e texto
\vspace{-1.9cm}

% Texto do capítulo
  
  Pretende-se com esta proposta de pesquisa investigar e elaborar aplicações em tempo real, 
  para a internet, utilizando a linguagem JavaScript no servidor Node.Js.
  
  Atualmente a internet tem crescido mais depressa que o rádio e a televisão. 
  Hoje a Internet, como mídia de comunicação, tem o mais amplo de todos os alcances do que as mídias 
  citadas anteriormente, conforme pode ser visto na pesquisa do IBGE de 2005 para 2011, 
  número de internautas cresce 143.8\%  e o de pessoas com celular, 107.2\%. 
  E para continuar a oferecer serviços e informações, com rapidez e até mesmo em tempo real, 
  é necessário se preocupar com a quantidade de milhões de usuários simultâneos, 
  que cresce exponencialmente, e vencer barreiras tecnológicas de escalabilidade e desempenho nos servidores.
  
  Segundo \citeonline{Tilkov:2002}, para resolver problemas de que lidam com múltiplas entradas e saídas \ac{E/S},
  como manipular múltiplas requisições de clientes em servidores, os programadores adotaram utilizar técnicas 
  de programação multithread (processamento simultâneo de um conjunto de tarefas), 
  ou técnicas de programação paralela, dividindo o processamento da aplicação em vários núcleos 
  dos processadores da \ac{CPU} ou até mesmo utilizando computação distribuída. 
  Este modelo de programação para atender múltiplas entradas e saídas é fácil de entender, 
  implementar e permite executar os processos de forma rápida e eficiente. 
  Porém este modelo apresenta algumas falhas, por exemplo, quando uma thread (conjunto de tarefas) 
  consome um recurso X de processamento ou operação de entrada e saída e em seguida o aplicativo executa uma nova 
  thread (conjunto de tarefas) que necessita consumir este mesmo recurso, 
  teremos um bloqueio o qual é necessário esperar a primeira thread (conjunto de tarefas) 
  terminar sua execução, liberar o recurso e então prosseguir com o processamento da segunda thread. 
  Como dito por \citeonline{Tilkov:2002} mesmo que muitos tenham obtido sucesso em usar 
  multithread (processamento simultâneo de um conjunto de tarefas) em aplicações, 
  não é fácil isolar e corrigir problemas como bloqueios e falhas, 
  proteger recursos compartilhados entre as threads (conjunto de tarefas). 
  Também perde-se o controle quando ocorre o desenvolvimento 
  multithreading (processamento simultâneo de um conjunto de tarefas) pois o sistema operacional é responsável 
  por decidir qual thread (conjunto de tarefas) será executada e por quanto tempo.''~\cite[p. 80]{Tilkov:2010}
  
  Em sistemas web desenvolvidos sob as plataformas tradicionais como JAVA, \ac{PHP}, .NET dentre outros 
  é necessário paralisar um processamento enquanto utiliza uma entrada e saída do servidor. 
  Essa paralisação é conhecida como um modelo bloqueante. Exemplificando este modelo, em um servidor Web 
  que cada processo é uma requisição de feita pelo usuário. Com o decorrer novos usuários realizam novas 
  requisições aumentando o processamento. No modelo bloqueante cada requisição é enfileirada e depois 
  processadas uma a uma. Enquanto uma requisição esta sendo processada as demais ficam em espera, 
  mantendo-se ociosas por um período indeterminado na fila.\cite{Pereira:2013}
  
  Com esta arquitetura tradicional, gasta-se muito tempo mantendo uma fila de espera com processos ociosos,
  tais como: envio de e-mails, consultas em banco de dados, leitura em disco que não liberam recursos enquanto
  não forem finalizadas. Com o aumento dos acessos ao sistema é necessário fazer uma atualização
  do hardware (equipamento).\cite{Pereira:2013}
  
  De acordo com \citeonline{Abernethy:2011}, explica que em linguagens como Java e \ac{PHP}, cada conexão cria-se uma 
  nova thread ( conjunto de tarefas ) com 2 MB de memória RAM. Se em um servidor possuir 8 GigaBytes de memória RAM, 
  teoricamente o número máximo de conexões concorrentes é aproximadamente 4.000 usuários. 
  Com o aumento da base de cliente, e claro, se quiser que o aplicativo web suporte mais usuários, é necessário 
  adicionar mais e mais servidores. Somando aos custos de adição de novos equipamentos, ha possíveis problemas 
  técnicos que devem ser considerados tal como o um usuário usar diferentes servidores para cada requisição, portanto, 
  os recursos devem estar compartilhados em todos os servidores. Por todas essas razões, o gargalo em toda a arquitetura 
  da aplicação web ( incluindo a velocidade de tráfego, velocidade do processador e velocidade da memória RAM) 
  estaria associado ao número máximo de conexões concorrentes que um servidor pode manipular.
  
  Portanto, observa-se que o escalonamento horizontal, adicionando novos servidores, além do custo altíssimo, 
  torna a arquitetura do sistema complexa pois será necessário acrescentar servidores de balanceamento, 
  rede estruturada da central de dados que seja capaz de suportar um alto tráfego e acompanhamento dos processos 
  do sistema de perto para que os bloqueios sejam consertados em tempo hábil. A utilização de escalonamento vertical, 
  ou melhor, atualização de hardware – colocando mais processadores ou memória - pode inviabilizar a arquitetura do 
  sistema, visto que ha uma barreira de hardware, mais especificamente, placas-mãe que não suportam mais de 8 slots 
  de memória ou determinados modelos de memória RAM, suporte a processadores com mais de 7 núcleos. 
  Além dessas limitações tecnológicas, ha o agravante do alto custo para atualizar este hardware. 
  Processadores com 7 núcleos são caros e dependendo dos casos é necessário trocar todo o equipamento - hardware - 
  para garantir o devido funcionamento dos componentes.
  
  Pelos problemas citados acima surge a necessidade de resolver este problema, 
  em nível de software, que permita receber um grande número de conexões simultâneas 
  nos servidores, capaz de ser escalável e consumir menores índices de memória RAM. 
  Um paradigma adotado para esta solução é a programação orientada a eventos, onde tudo gira em torno de eventos, 
  indicando que exite um produtor do evento e um consumidor daquele evento.\cite{Junior:2012}
  
  O JavaScript, linguagem de programação, fornece o modelo de eventos assíncronos, funções anônimas e callbacks. 
  Como \citeonline{Junior:2012} exemplificou, um programa assíncrono ao fazer uma requisição a um banco de dados especifica 
  o que deve ser feito com os resultados do banco de dados. 
  Este programa não espera a finalização da requisição e processa outras atividades. 
  Apenas quando o resultado da requisição é retornado do banco de dados, a codificação para manipular os estes dados 
  é executado. A esta lógica de programação, executada após a fim da requisição, dá-se ao nome de callback.\cite[p. 2]{Junior:2012} 
  
  A partir dessa necessidade surge o ambiente de desenvolvimento Node.Js, 
  que é melhor descrito por \citeonline{Junior:2012} como uma plataforma cujo o objetivo é a fácil 
  construção de rápidas e escaláveis aplicações de rede. Para isto o Node.Js emprega orientação 
  a eventos utilizando o JavaScript Engine V8 do Google, operações de entradas e saídas em eventos (assíncronos) 
  e não bloqueantes. Abenerthy cita que ao invés de criar novas threads (conjunto de tarefas) no sistema operacional 
  para cada conexão e alocar a memória RAM que acompanha essas threads (conjunto de tarefas), 
  cada conexão dispara um evento executado no processo do motor Node.Js. 
  O Node.Js afirma que ele nunca irá ter bloqueios ou impasses, já que bloqueios não é uma característica 
  da sua plataforma mesmo em processamento de entradas e saídas.. Node.Js afirma que um servidor pode suportar 
  dezenas de milhares de conexões simultâneas.\cite{Abernethy:2011}
  
  Por fim, busca-se com o Node.Js, o qual será a base para esta proposta de pesquisa, 
  demonstrar uma aplicação Web capaz de mostrar, em tempo real, a localização de dispositivos móveis através 
  das coordenadas de latitude e longitude utilizando o paradigma de orientação a eventos, 
  com alta concorrência de conexões.
  
\section{Motivação}
\label{motivacao}
  
  Com a chegada de aplicativos web integrados a dispositivos móveis há a necessidade de se ter
  uma arquitetura capaz de suportar milhares de usuários e que ela atenda a todos os graus de satisfação deles. 
  No repositório do Node.Js possui uma lista de projetos, aplicações e empresas que utilizam o Node.Js. 
  Dentre estes items pode-se destacar duas empresas, o LinkdIn que utiliza o ambiente Node.JS para aplicações 
  móveis e o PayPal famoso gateway - ponte - de pagamento.
  
\section{Objetivos}
\label{objetivos}

\subsection{Objetivos Geral}

  Pretende-se com esta pesquisa investigar, comparar e demonstrar a capacidade 
  do ambiente Node.Js de processar e responder milhares de requisições comparando-o com um ambiente Python.
  
  Para isso tem-se dois aplicativos desenvolvidos como uma API RESTFul provendo as operações
  básicas como \ac{CRUD} de uma lista de contatos.
  
  De antemão não será desenvolvido um sistema complexo como os já existentes no mercado.
  
\subsection{Objetivos específicos}

  Os objetivos específicos deste trabalho são:
  
    \begin{compactitem}
      \item[a)] Estudo sobre ambiente de desenvolvimento Node.J
      \item[b)] Apresentar a metodologia de desenvolvimento da arquitetura REST em Node.Js;
      \item[c)] Utilização do protótipo desenvolvido.
      \item[d)] Testes de cargas em aplicações web.
      \item[e)] Apresentar os resultados alcançados através dos testes realizados.
    \end{compactitem}
  
\section{Problema}
\label{problema}

  No cenário atual, em termos de desenvolvimento web com as linguagens de programação tradicionais, 
  para garantirmos disponibilidade das informações a milhares de usuário é necessário um conjunto 
  de ferramentas e técnicas. Como exemplo tem-se o vídeo da palestra Usando Django para atender 12 milhões de usuários 
  apresentada por Rômulo Jales e Victor Pantoja no evento da Python Brasil 9.
  Nesta palestra foi apresentada a arquitetura utilizada pelo portal 
  Globo Esporte\footnote{Sítio www.globoesporte.com.br} utilizando várias técnicas 
  para ganhar desempenho, tais como, servir páginas HTML estáticas através do 
  NGINX\footnote{Nginx [engine x] é um servidor HTTP e proxy reverso}, gravar as páginas geradas em 
  disco, utilizar 
  SSI\footnote{SSI são diretivas que são colocadas em páginas HTML, enquanto as páginas estão sendo servidas} 
  ao invés de chamadas 
  Ajax\footnote{Ajax Javasvript assíncrono e xml, utilizado para atualizar partes da página web sem recarrega-la}, 
  ter um sistema de cache de objeto distribuído em memória   como o Memcached e um acelerador de aplicações web
  cache HTTP e proxy reverso tal como Varnish, minificar  arquivos CSS e JavaScript, utilizar CSS Sprite, utilizar 
  compactação nos arquivos servidos através de, dentre outras técnicas.
  
  Para alcançar este número de usuários suportados no portal é necessário ter uma equipe 
  capacitada para realizar técnicas de programação eficazes, analise e configuração de ferramentas de cache 
  disponíveis no mercado e uma central de dados capacitada. O custo para manter esse eco sistema funcionando,
  provavelmente é alto.
  
  Mediante a esta análise surge a proposta de utilizar o ambiente de programação Node.Js para comprovar sua 
  capacidade de aceitar milhares requisições de usuários utilizando o paradigma de programação orientada a eventos 
  no servidor.
  
\section{Organização}
\label{organizacao}  

  Para contextualizar o leitor, o Capítulo \ref{ambiente-node-js} aborda o referêncial teórico e fontes de estudos utilizados para iniciar com o ambiente 
  Node.Js. Já o Capítulo \ref{desenvolvimento-prototipos} compreende a lista de requisitos e especificações do protótipo a ser construído e o
  desenvolvimento. O Capítulo \ref{experimentos-resultados} descreve os testes realizados e os resultados obtidos. 
  Por fim, o Capítulo \ref{}, conclui o trabalho acadêmico.