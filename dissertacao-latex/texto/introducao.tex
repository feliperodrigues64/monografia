% Nome do capítulo
\chapter{Introdução}
% Label para referenciar
\label{introducao}

% Diminuir espaçamento entre título e texto
\vspace{-1.9cm}

% Texto do capítulo
  
  A Internet como mídia de comunicação, possui o mais amplo de todos os alcances. 
  Conforme pode ser visto na pesquisa do \citeonline{IBGE:2013}. Entre 2005 para 2011, 
  o número de internautas cresceu 143.8\%  e o de pessoas com celular, 107.2\%.
  Com este aumento é necessário se preocupar com a quantidade de milhões de usuários simultâneos,
  que cresce exponencialmente ao longo dos anos, vencendo barreiras tecnológicas, tais como 
  escalabilidade e desempenho para oferecer serviços e informações com rapidez e até mesmo em tempo
  real para o usuário final.
  
  Esta falta de desempenho dos aplicativos Web é influenciada pelo consumo de memória RAM. \citeonline{Abernethy:2011} 
  explica que em linguagens como Java e PHP, a cada conexão cria-se uma nova \textit{thread} com 2 MB de memória RAM. 
  Nesta linha de raciocínio, o autor explica que com um servidor de 8 gigabytes de memória RAM suportaria
  aproximadamente 4.000 usuários conectados.
  
  Outro fator relacionado à queda de desempenho dos aplicativos é a maneira como linguagens de desenvolvimento de sistemas como JAVA, PHP,
  .NET manipulam as requisições e tratam as operações de entrada e saída na aplicação.   
  Nestas plataformas de desenvolvimento tradicionais um processamento é interrompido quando ocorre uma operação de entrada e saída.
  Essa interrupção, conhecida como modelo bloqueante, enfileira cada requisição e posteriormente processa uma a uma. Desta forma, 
  enquanto uma requisição está sendo processada, as demais entram em estado de espera ficando ociosas
  por períodos indeterminados\cite{Pereira:2013}.
  
  Através deste problema, surgiu a necessidade de melhorar em nível de desenvolvimento de software o grande número de conexões 
  simultâneas, escalabilidade e baixo consumo de memória RAM de um aplicativo web\cite{Oliveira:2012}.
  
  Pretende-se com esta proposta de pesquisa investigar e elaborar aplicações para a Web 
  utilizando o servidor Node.Js e comparar a sua performance em relação ao desenvolvimento
  de aplicativos tradicionais.
  
 
  
%  Com esta arquitetura tradicional, gasta-se muito tempo mantendo uma fila de espera com processos ociosos,
%  tais como: envio de e-mails, consultas em banco de dados, leitura em disco que não liberam recursos enquanto
%  não forem finalizadas. Com o aumento dos acessos ao sistema é necessário fazer uma atualização
%  do hardware (equipamento).\cite{Pereira:2013}
  
  
%  \citeonline{Abernethy:2011}, explica que em linguagens como Java e PHP, cada conexão cria-se uma 
%  nova \textit{thread} com 2 MB de memória RAM. Se em um servidor possuir 8 GigaBytes de memória RAM, 
%  teoricamente o número máximo de conexões concorrentes é aproximadamente 4.000 usuários. 
%  Com o aumento da base de cliente, e claro, se quiser que o aplicativo web suporte mais usuários, é necessário 
%  adicionar mais e mais servidores. Como foi descrito, o gargalo em toda a arquitetura 
%  da aplicação web ( incluindo a velocidade de tráfego, velocidade do processador e velocidade da memória RAM) 
%  estaria associado ao número máximo de conexões concorrentes que um servidor pode manipular.
  
%  Portanto, observa-se que o escalonamento horizontal, adicionando novos servidores, além do custo altíssimo, 
%  torna a arquitetura do sistema complexa pois será necessário acrescentar servidores de balanceamento, 
%  rede estruturada da central de dados que seja capaz de suportar um alto tráfego e acompanhamento dos processos 
%  do sistema de perto para que os bloqueios sejam consertados em tempo hábil. A utilização de escalonamento vertical, 
%  ou melhor, atualização de hardware – colocando mais processadores ou memória - pode inviabilizar a arquitetura do 
%  sistema, visto que há uma barreira de hardware, mais especificamente, placas-mãe que não suportam mais de 8 slots 
%  de memória ou determinados modelos de memória RAM, suporte a processadores com mais de 7 núcleos. 
%  Além dessas limitações tecnológicas, ha o agravante do alto custo para atualizar este hardware. 
%  Processadores com 7 núcleos são caros e dependendo dos casos é necessário trocar todo o equipamento - hardware - 
%  para garantir o devido funcionamento dos componentes.
  

  
\section{Motivação}
\label{motivacao}
  
  Com alto crescimento da internet, surgiu a necessidade de que as aplicações para web sejam capazes de suportar milhares de usuários,
  escaláveis e de baixa arquitetura. O Node.Js, objeto de estudo, entra no cenário tecnológico como plataforma de desenvolvimento rápido
  de aplicações de rede escaláveis.
  
\section{Objetivos}
\label{objetivos}


\subsection{Objetivo Gerais}

  Pretende-se com esta pesquisa investigar e comparar a capacidade 
  do ambiente Node.Js de processar e responder requisições em relação a um aplicativo desenvolvido 
  no paradigma tradicional.
 
  
\subsection{Objetivos específicos}

  Os objetivos específicos deste trabalho são:
  
    \begin{compactitem}
      \item[a)] Verificar o tempo médio de resposta gasto ao requisições;
      \item[b)] Obter resultados do número de sucessos obtidos nas requisições;
      \item[c)] Quantificar o desempenho entre cada teste;
    \end{compactitem}
  
  
\section{Organização}
\label{organizacao}  

  Para contextualizar o leitor, o Capítulo \ref{ambiente-node-js} aborda o referencial teórico e as fontes de estudos utilizados para iniciar com o ambiente 
  Node.Js. O capítulo \ref{metodologia} apresenta a metodologia utilizada no decorrer deste trabalho. O Capítulo \ref{desenvolvimento-prototipos} compreende a lista de requisitos e especificações do protótipo a ser construído e o
  desenvolvimento. O Capítulo \ref{experimentos-resultados} descreve os testes realizados e os resultados obtidos. 
  Por fim, o Capítulo \ref{conslusao-e-trabalhos-furutos}, conclui o trabalho acadêmico.