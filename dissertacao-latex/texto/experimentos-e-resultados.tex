% Nome do capítulo
\chapter{Experimentos e resultados}
% Label para referenciar
\label{experimentos-resultados}

% Diminuir espaçamento entre título e texto
\vspace{-1.9cm}

% Texto do capítulo

 Neste caítulo vamos abordar o ambiente de testes para os nossos protótipos e apresentar 
 o plano de testes utilizado.
 
 Após o detalhamento do plano de testes serão apresentados ao leitor os resultados obtidos e análise dos dados.

\section{Ambiente de testes}
\label{ambientedetestes}

  Para efetuar os testes, os protótipos tiveram de ser colocados em uma ambiente de testes. 
  Para os dois protótipos, hospedamos a aplicação em uma VPS\footnote{http://digitalocean.com} da empresa Digital Ocean.
  
  O ambiente é composto pelos seguintes componentes de hardware conforme descrito nas tabelas\footnote{https://www.digitalocean.com/pricing/}
  de precificação do produto.
  
  \begin{table}[H]
    \centering
    \footnotesize
    % Alterar espaçamentos antes e depois do caption
    \setlength{\abovecaptionskip}{0pt}
    \setlength{\belowcaptionskip}{0pt}
    % Caption
    \caption[Componentes da VPS]{Componentes da VPS}
    \label{tab:components-digital-ocean-vps}
    % Conteúdo da tabela
    \begin{tabular}{c|c|}
      \hline \hline
      Componente  &	Descrição \\
      \hline \hline
      Memória Ram & 512 MegaBytes \\
      Processador & 1 núcleo. \\
      Espaço em disco & 20 GigaBytes \ac{SSD}. \\
      Transferência em rede & 1 TeraByte. \\
      \hline \hline
    \end{tabular}
    % Fonte
    \captionfont{\small{\textbf{\\Fonte: Digital Ocean Pricing <https://www.digitalocean.com/pricing/> acesso 05. NOV. 2014}}}
  \end{table}

  Para os dois protótipos busca-se manter o máximo de igualdade entre os serviços executados, porém, por ser 
  tecnologias diferentes o modo de \textit{deploy} também são diferentes adicionando ou não novos serviços. 
  As informações dos serviços executados está listada de acordo com o software de monitoramento\footnote{http://newrelic.com}
  da empresa New Relic.
  
  Apresenta-se os serviços em execução \ref{tab:services-in-api-django}do protótipo feito em Django sem clientes conectados.
  Este ambiente é composto por um servidor Nginx que suporta através de um \textit{proxy} a execução do \textit{gunicorn},
  que nada mais é a execução do framework Django, o banco de dados Postgres 9.1 e o processo supervisord responsável por
  verificar se a aplicação esta ativa e caso ocorra alguma falha é capaz de restartar a aplicação.
  
  \begin{table}[H]
    \centering
    \footnotesize
    % Alterar espaçamentos antes e depois do caption
    \setlength{\abovecaptionskip}{0pt}
    \setlength{\belowcaptionskip}{0pt}
    % Caption
    \caption[Serviços executados na API Django]{Serviços executados na API Django}
    \label{tab:services-in-api-django}
    % Conteúdo da tabela
    \begin{tabular}{c|c|c|c|c|}
      \hline \hline
      Processo  & 	CPU \% &	Memória \\
      \hline \hline
      gunicorn &	0.0\% &		104 MB \\
      postgres &	0.0\% &		19.3 MB \\
      supervisord &	0.0\% &		11.5 MB \\
      nginx &		0.0\% &		7.26 MB \\
      getty &		0.0\% &		5.69 MB \\
      udevd &		0.0\% &		4.05 MB \\
      nrsysmond &	0.1\% &		3.93 MB \\
      console-kit-dae &	0.0\% &		3.84 MB \\
      polkitd &	 	0.0\% &		2.98 MB \\
      ntpd &		0.0\% &		2.34 MB \\
      rsyslogd &	0.0\% &		1.47 MB \\
      nginx &		0.0\% &		1.38 MB \\
      sshd &		0.0\% &		1.22 MB \\
      dbus-daemon &	0.0\% &		1.13 MB \\
      cron &		0.0\% &		1.03 MB \\
      exim4 &		0.0\% &		992 KB \\
      init &		0.0\% &		820 KB \\
      acpid &		0.0\% &		656 KB \\
      atd &		0.0\% &		156 KB \\
      su &		0.0\% &		0 \\
      bash &		0.0\% &		0 \\
      \hline \hline
    \end{tabular}
    % Fonte
    \captionfont{\small{\textbf{\\Fonte: Autor}}}
  \end{table}

  Para o segundo protótipo, com Node.Js, possuimos os seguintes serviços executados \ref{tab:services-in-api-node} 
  também sem nenhum cliente conectado. Este ambiente é composto por um servidor Nginx que suporta através 
  de um \textit{proxy}.
  
  \begin{table}[H]
    \centering
    \footnotesize
    % Alterar espaçamentos antes e depois do caption
    \setlength{\abovecaptionskip}{0pt}
    \setlength{\belowcaptionskip}{0pt}
    % Caption
    \caption[Serviços executados na API Node]{Serviços executados na API Node}
    \label{tab:services-in-api-node}
    % Conteúdo da tabela
    \begin{tabular}{c|c|c|c|c|}
      \hline \hline
      Processo  & 	CPU \% &	Memória \%  \\
      \hline \hline
      \hline \hline
    \end{tabular}
    % Fonte
    \captionfont{\small{\textbf{\\Fonte: Autor}}}
  \end{table}
  
\subsection{Primeira subseção}
  
  As enumerações devem ser geradas usando o pacote \textit{compactitem}. Cada item deve terminar com um ponto final.
  Abaixo um exemplo de enumeração é apresentado:

    \begin{compactitem}
      \item[a)] Coletar e analisar.
      \item[b)] Configurar e simular.
      \item[c)] Definir a metodologia.
      \item[d)] Avaliar o desempenho.
      \item[e)] Analisar e avaliar características.
    \end{compactitem}

\section{Segunda seção}

  Para referenciar um capítulo, seção ou subseção basta definir um label para o mesmo e usar o comando ref para referênciá-lo
  no texto. Exemplo: Como pode ser visto no Capítulo \ref{cap1} ou na Seção \ref{secao1}.